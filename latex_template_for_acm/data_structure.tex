\subsection{RMQ}

\begin{lstlisting}
int mx[maxn][20];
int f[maxn], num[maxn];

void build(){
    for(int i = 1; i <= n; i++)
        mx[i][0] = f[i];
    for(int j = 1; (1<<j)<= n; j++)
        for(int i = 1; i + (1<<j)- 1<= n; i++)
            mx[i][j] = max(mx[i][j-1], mx[i + (1<<(j-1))][j - 1]);
}

int RMQ(int s, int v){
    if(s > v)return 0;
    int k = (int)(log(v - s + 1)/ log(2));
    return max(mx[s][k], mx[v - (1<<k) + 1][k]);
}	
\end{lstlisting}

二维RNQ
\begin{lstlisting}
int mp[maxn][maxn];
int mx[maxn][maxn][10], mi[maxn][maxn][10];

void build(){
    for (int i = 1; i <= n; i++)
        for (int j = 1; j <= n; j++)
            mx[i][j][0] = mi[i][j][0] = mp[i][j];
    for (int i = 1; i <= n; i++)
        for (int j = 1; (1<<j)<= n; j++)
        for (int k = 1; k <= n; k++){
            mx[i][k][j] = max(mx[i][k][j-1], mx[i][k+(1<<(j-1))][j-1]);
            mi[i][k][j] = min(mi[i][k][j-1], mi[i][k+(1<<(j-1))][j-1]);
        }
}

int query(int x, int y){
    int k = (int)(log(b)/log(2));
    int maxx = -inf, minx = inf;
    for(int i = x; i <= x+b-1; i++){
        maxx = max(max(mx[i][y][k], mx[i][y - (1<<k) + b][k]), maxx);
        minx = min(min(mi[i][y][k], mi[i][y - (1<<k) + b][k]), minx);
    }
    return maxx - minx;
}
\end{lstlisting}

01 RMQ

\begin{lstlisting}
#include <bits/stdc++.h>
using namespace std;
#define PB push_back
#define ZERO (1e-10)
#define INF (1<<29)
#define CL(A,I) (memset(A,I,sizeof(A)))
#define DEB printf("DEB!\n");
#define D(X) cout<<"  "<<#X": "<<X<<endl;
#define EQ(A,B) (A+ZERO>B&&A-ZERO<B)
typedef long long ll;
typedef long double ld;
typedef pair<ll,ll> pll;
typedef vector<int> vi;
typedef pair<int,int> ii;
typedef vector<ii> vii;
#define IN(n) int n; cin >> n;
#define FOR(i, m, n) for (int i(m); i < n; i++)
#define REP(i, n) FOR(i, 0, n)
#define F(n) REP(i, n)
#define FF(n) REP(j, n)
#define FT(m, n) FOR(k, m, n)
#define aa first
#define bb second
void ga(int N,int *A){F(N)scanf("%d",A+i);}
#define LG (20)
#define MX (1<<LG)
#define BT (12) //K =~= log(N)/2
#define SZ (MX/BT+1)
#define P2(v) (!(v&(v-1)))
//structure for RMQ which differs exactly by 1
struct RMQ{
    /*
     * T[b][i][j]: Smallest element on interval i-j, in gradient of shape "b"
     * G: Precalculate logarithms
     * B: Bitmask of gradients of each interval of size BT
     * I: Lowest element on each little interval
     * E: Value of lowest element on each interval
     * dp: RMQ: Sparse table
     */
    char T[1<<BT][BT][BT],G[MX];
    short B[SZ];
    int X,I[SZ],E[SZ],O=-1,dp[SZ][LG];
    //Precaltucaliton of table T: easy simulation of process for each possible bitmask
    //Complexity is B^2*2^B
    void tb(int B){
        F(1<<B)FF(B+1){
            int d=0,b=0,I=j;
            FT(j,B+1){
                T[i][j][k]=I;
                d+=i&1<<k?1:-1;
                if(b>d)b=d,I=k+1;
            }
        }
    }
    //Makes bitmask of an interval
    //It is just BT-1 (not BT) since we are interested in differences only
    int bt(int*A){
        int b=0;
        //Check difference and shift it to i-th position
        F(BT-1)b|=(A[i+1]>A[i])<<i;
        return b;
    }
    //Body of function which calls other functions - setting proper values to arrays
    void ini(int N,int*A){
        if(!X++){tb(BT-1);FT(1,MX)G[k]=O+=P2(k);}
        F(BT*2)A[N+i]=A[N+i-1]+1;
        F(N/BT+1)E[i]=*min_element(A+i*BT,A+i*BT+BT),I[i]=min_element(A+i*BT,A+i*BT+BT)-A,B[i]=bt(A+i*BT);
        ST(N/BT+1);
    }
    //Function which calculates SPARSE TIBLE (of sizeN/BT)
    void ST(int N){
        F(N)dp[i][0]=i;
        //Precalcute so that there is index to array "E".
        //Length 2^i: Minimal value from index i to i+2^i-1
        //Counting intervals from smaller to bigger
        FT(1,k-(1<<k)+N+1)F(N+1-(1<<k))
            if(E[dp[i][k-1]]<E[dp[i+(1<<(k-1))][k-1]])
                dp[i][k]=dp[i][k-1];
            else dp[i][k]=dp[i+(1<<(k-1))][k-1];
    }
    //Returns minimal value on L - R interval
    //j: logarithm of interval-size
    //Taking interval from L to L+2^j-1 and from R-2^j+1 to R, and takes
    //minimum of thos (it will surely cover the whole interval)
    int QQ(int L,int R,int*A){
        int j=G[R-L+1];return A[dp[L][j]]<=A[dp[R-(1<<j)+1][j]]?dp[L][j]:dp[R-(1<<j)+1][j];
    }
    inline int sc(int a){return a-a%BT;}
    //Query to interval of size BT
    //Determines borders of interval
    void qp(int L,int R,int&b,int&J,int*A){
        int u=T[B[L/BT]][L%BT][R%BT];
        if(b>A[sc(L)+u])b=A[J=sc(L)+u];
    }
    //Function which solves RMQ on interval
    //It solves it in multiple steps: it solves it on the BIG interval + it solves the
    //borders of it
    //All call are O(1)
    int qy(int L,int R,int*A){
        int a,b=INF,J=-1;
        if(L/BT==R/BT)return qp(L,R,b,J,A),J;
        if(L%BT)a=L,L=sc(L)+BT,qp(a,L-1,b,J,A);
        if(R%BT+1!=BT)a=R,R=sc(R)-1,qp(R+1,a,b,J,A);
        if(L<=R&&E[a=QQ(L/BT,R/BT,E)]<b)b=a,J=I[a];
        return J;
    }
};
struct LCA{
    /*
     * R: RMQ for +/-1
     * g: Edges
     * l: Size of E
     * E: Flattened tree in rder of Euler Tour (with nesting)
     * L: Depth of node
     * H: Index of first position of node int 'E'
     * D: Depth of node. With parallel to 'E'
     */
    RMQ R;
    vi g[MX];
    int l,E[MX],L[MX],H[MX],D[MX];
    //dfs for euler tour: Node, Parent, Depth, Indext to E
    void dfs(int u,int p,int d,int&l){
        H[u]=l,L[u]=d++,E[l]=u,D[l++]=L[u];
        for(auto&h:g[u])if(h^p)dfs(h,u,d,l),E[l]=u,D[l++]=L[u];
    }
    //Init - nothing interesting (just calls)
    void pre(int N,int r=0){dfs(r,r,0,l=0),R.ini(l,D);}
    //Returns LCA: Returns value of E on index given by RMQ
    int lca(int a,int b){a=H[a],b=H[b];if(a>b)swap(a,b);return E[R.qy(a,b,D)];}
    //Adds edge
    void ADD(int A,int B){g[A].PB(B),g[B].PB(A);}
    //Clears graph
    void CLR(){for(auto&h:g)h.clear();}
}L;
\end{lstlisting}


\subsection{线段树}

\begin{lstlisting}
namespace sgt {
    long long chg[1000010];
    int son[1000010][2],cnt,root;
    void build(int &x,int l,int r) {
        x=++cnt;
        if(l==r) return ;
        int mid=(l+r)>>1;
        build(son[x][0],l,mid);
        build(son[x][1],mid+1,r);
    } void update(int a,int b,int k,int l,int r,long long v) {
        if(a>b || l>r)return;
        if(a<=l && b>=r)chg[k]+=v;
        else {
            int mid=(l+r)>>1;
            if(b<=mid)update(a,b,son[k][0],l,mid,v);
            else if(a>mid)update(a,b,son[k][1],mid+1,r,v);
            else update(a,mid,son[k][0],l,mid,v),update(mid+1,b,son[k][1],mid+1,r,v);
        }
    } long long query(int a,int k,int l,int r,long long v) {
        if(l==r)return v+chg[k]+val[a];
        int mid=(l+r)>>1;
        if(a<=mid)return query(a,son[k][0],l,mid,v+chg[k]);
        else return query(a,son[k][1],mid+1,r,v+chg[k]);
    }
}
\end{lstlisting}


\subsection{LCA}
垃圾lca
\begin{lstlisting}
int lca(int x, int y){
    if(depth[x] > depth[y]) swap(x, y);
    while(depth[x] != depth[y]){
        y = fa[y];
    } while(x!=y){
        x = fa[x]; y = fa[y];
    } return x;
}
\end{lstlisting}

倍增LCA
\begin{lstlisting}
void dfs(int x){
    depth[x] = depth[fa[x][0]]+1;
    for(int i = 0; fa[x][i]; i++) 
        fa[x][i+1] = fa[fa[x][i]][i];
    for(int i = head[x]; i != -1; i=edge[i].next){
        int to = edge[i].to;
        if(!depth[to]){
            fa[to][0] = x;
            dis[to] = dis[x] + edge[i].cost;
            dfs(to);
        }
    }
}

int Lca(int x, int y){
    if(depth[x] > depth[y]) swap(x, y);
    for(int i = 20; i >= 0; i--)
        if(depth[fa[y][i]] >= depth[x])
            y = fa[y][i];
    if(x == y) return y;
    for(int i = 20; i >= 0; i--){
        if(fa[y][i] != fa[x][i]){
            y = fa[y][i];
            x = fa[x][i];
        }
    }
    return fa[x][0];
}
\end{lstlisting}


\subsection{树状数组}

{\bfseries 逆序对数}

{\begin{lstlisting}
#include<bits/stdc++.h>
#define ll long long
const int maxn = 5e5+10;
using namespace std;
int bit[maxn], n;
struct Num {
    int val,pos;
    bool operator <(const Num &a)const {
        return val < a.val;
    }
}num[maxn];

int lowbit(int x){ return x&(-x); }
int sum(int i) {
    int res = 0;
    while(i > 0) {
        res += bit[i];
        i -= lowbit(i);
    } return res;
}

void add(int i, int x) {
    while(i <= n) {
        bit[i] += x;
        i += lowbit(i);
    }
}

int main(){
    while(~scanf("%d",&n) && n){
        memset(bit, 0, sizeof bit);
        ll ans = 0;
        for(int i = 0; i < n; i++){
            scanf("%d",&num[i].val);
            num[i].pos = i+1;
        } sort(num, num + n);
        for(int i = 0; i < n; i++){
            int tmp = sum(num[i].pos);//`小于他的个数`
            ans += i - tmp;
            add(num[i].pos, 1);
        } printf("%lld\n",ans);
    } return 0;
}
\end{lstlisting}}

二维树状数组

\begin{lstlisting}
int lb(int x){
    return x&(-x);
}

int sum(int x, int y){
    int res = 0;
    for(int i = x; i > 0; i -= lb(i))
        for(int j = y; j > 0; j -= lb(j))
            res += bit[i][j];
    return res;
}

void add(int x, int y, int k){
    for(int i = x; i < maxn; i += lb(i))
        for(int j = y; j < maxn; j += lb(j))
            bit[i][j] += k;
}
\end{lstlisting}




