\subsection{bitset}

bitset可以优化背包,竞赛图之类的东西。

\begin{lstlisting}
bitset<17>BS;
BS[1] = BS[7] = 1;
cout<<BS._Find_first()<<endl; // prints 1

BS[1] = BS[7] = 1;
cout<<BS._Find_next(1)<<','<<BS._Find_next(3)<<endl; // prints 7,7

for(int i=BS._Find_first();i< BS.size();i = BS._Find_next(i))
    cout<<i<<endl;
\end{lstlisting}

\subsection{笛卡尔树}

\begin{lstlisting}
int build(){
    int tp=0;
    for (int i = 1; i <= n; i++) l[i]=r[i]=vis[i]=0;
    for (int i = 1; i <= n; i++) {
        int k = tp;
        while(k > 0&&a[stk[k-1]]<a[i]) --k;
        if(k) r[stk[k-1]] = i;
        if(k < tp) l[i] = stk[k];
        stk[k++] = i;
        tp = k;
    } for (int i = 1; i <= n; i++) {
        vis[l[i]] = vis[r[i]] = 1;
    } for (int i = 1; i <= n; i++) if(!vis[i]) return i;
}
\end{lstlisting}

\subsection{昌昌的快读}

\begin{lstlisting}
namespace fastIO{
#define BUF_SIZE 100000
#define OUT_SIZE 100000
#define ll long long
  //fread->read
  bool IOerror=0;
  inline char nc(){
    static char buf[BUF_SIZE],*p1=buf+BUF_SIZE,*pend=buf+BUF_SIZE;
    if (p1==pend){
      p1=buf; pend=buf+fread(buf,1,BUF_SIZE,stdin);
      if (pend==p1){IOerror=1;return -1;}
      //{printf("IO error!\n");system("pause");for (;;);exit(0);}
    }
    return *p1++;
  }
  inline bool blank(char ch){return ch==' '||ch=='\n'||ch=='\r'||ch=='\t';}
  inline void read(int &x){
    bool sign=0; char ch=nc(); x=0;
    for (;blank(ch);ch=nc());
    if (IOerror)return;
    if (ch=='-')sign=1,ch=nc();
    for (;ch>='0'&&ch<='9';ch=nc())x=x*10+ch-'0';
    if (sign)x=-x;
  }
  inline void read(ll &x){
    bool sign=0; char ch=nc(); x=0;
    for (;blank(ch);ch=nc());
    if (IOerror)return;
    if (ch=='-')sign=1,ch=nc();
    for (;ch>='0'&&ch<='9';ch=nc())x=x*10+ch-'0';
    if (sign)x=-x;
  }
  inline void read(double &x){
    bool sign=0; char ch=nc(); x=0;
    for (;blank(ch);ch=nc());
    if (IOerror)return;
    if (ch=='-')sign=1,ch=nc();
    for (;ch>='0'&&ch<='9';ch=nc())x=x*10+ch-'0';
    if (ch=='.'){
      double tmp=1; ch=nc();
      for (;ch>='0'&&ch<='9';ch=nc())tmp/=10.0,x+=tmp*(ch-'0');
    }
    if (sign)x=-x;
  }
  inline void read(char *s){
    char ch=nc();
    for (;blank(ch);ch=nc());
    if (IOerror)return;
    for (;!blank(ch)&&!IOerror;ch=nc())*s++=ch;
    *s=0;
  }
  inline void read(char &c){
    for (c=nc();blank(c);c=nc());
    if (IOerror){c=-1;return;}
  }
  //fwrite->write
  struct Ostream_fwrite{
    char *buf,*p1,*pend;
    Ostream_fwrite(){buf=new char[BUF_SIZE];p1=buf;pend=buf+BUF_SIZE;}
    void out(char ch){
      if (p1==pend){
        fwrite(buf,1,BUF_SIZE,stdout);p1=buf;
      }
      *p1++=ch;
    }
    void print(int x){
      static char s[15],*s1;s1=s;
      if (!x)*s1++='0';if (x<0)out('-'),x=-x;
      while(x)*s1++=x%10+'0',x/=10;
      while(s1--!=s)out(*s1);
    }
    void println(int x){
      static char s[15],*s1;s1=s;
      if (!x)*s1++='0';if (x<0)out('-'),x=-x;
      while(x)*s1++=x%10+'0',x/=10;
      while(s1--!=s)out(*s1); out('\n');
    }
    void print(ll x){
      static char s[25],*s1;s1=s;
      if (!x)*s1++='0';if (x<0)out('-'),x=-x;
      while(x)*s1++=x%10+'0',x/=10;
      while(s1--!=s)out(*s1);
    }
    void println(ll x){
      static char s[25],*s1;s1=s;
      if (!x)*s1++='0';if (x<0)out('-'),x=-x;
      while(x)*s1++=x%10+'0',x/=10;
      while(s1--!=s)out(*s1); out('\n');
    }
    void print(double x,int y){
      static ll mul[]={1,10,100,1000,10000,100000,1000000,10000000,100000000,
        1000000000,10000000000LL,100000000000LL,1000000000000LL,10000000000000LL,
        100000000000000LL,1000000000000000LL,10000000000000000LL,100000000000000000LL};
      if (x<-1e-12)out('-'),x=-x;x*=mul[y];
      ll x1=(ll)floor(x); if (x-floor(x)>=0.5)++x1;
      ll x2=x1/mul[y],x3=x1-x2*mul[y]; print(x2);
      if (y>0){out('.'); for (size_t i=1;i<y&&x3*mul[i]<mul[y];out('0'),++i); print(x3);}
    }
    void println(double x,int y){print(x,y);out('\n');}
    void print(char *s){while (*s)out(*s++);}
    void println(char *s){while (*s)out(*s++);out('\n');}
    void flush(){if (p1!=buf){fwrite(buf,1,p1-buf,stdout);p1=buf;}}
    ~Ostream_fwrite(){flush();}
  }Ostream;
  inline void print(int x){Ostream.print(x);}
  inline void println(int x){Ostream.println(x);}
  inline void print(char x){Ostream.out(x);}
  inline void println(char x){Ostream.out(x);Ostream.out('\n');}
  inline void print(ll x){Ostream.print(x);}
  inline void println(ll x){Ostream.println(x);}
  inline void print(double x,int y){Ostream.print(x,y);}
  inline void println(double x,int y){Ostream.println(x,y);}
  inline void print(char *s){Ostream.print(s);}
  inline void println(char *s){Ostream.println(s);}
  inline void println(){Ostream.out('\n');}
  inline void flush(){Ostream.flush();}
#undef ll
#undef OUT_SIZE
#undef BUF_SIZE
};
using namespace fastIO;
\end{lstlisting}


\subsection{快速计算1-n质数个数}

\begin{lstlisting}
namespace pcf {
#define clr(ar) memset(ar,0,sizeof(ar))
#define chkbit(ar,i) (((ar[(i)>>6])&(1<<(((i)>>1)&31))))
#define setbit(ar,i) (((ar[(i)>>6])|=(1<<(((i)>>1)&31))))
#define isprime(x) (((x)&&((x)&1)&&(!chkbit(ar,(x))))||((x)==2))
const int MAXN=100;
const int MAXM=100010;
const int MAXP=666666;
const int MAX=10000010;
    ll dp[MAXN][MAXM];
    unsigned int ar[(MAX>>6)+5]={0};
    int len=0,primes[MAXP],counter[MAX];
    void Sieve() {
        setbit(ar,0),setbit(ar,1);
        for(int i=3;(i*i)<MAX;i++,i++) {
            if(!chkbit(ar,i)) {
                int k=i<<1;
                for(int j=(i*i);j<MAX;j+=k) setbit(ar,j);
            }
        } for(int i=1;i<MAX;i++) {
            counter[i]=counter[i-1];
            if(isprime(i)) primes[len++]=i,counter[i]++;
        }
    }
    void init() {
        Sieve();
        for(int n=0;n<MAXN;n++) {
            for(int m=0;m<MAXM;m++) {
                if(!n) dp[n][m]=m;
                else dp[n][m]=dp[n-1][m]-dp[n-1][m/primes[n-1]];
            }
        }
    }
    ll phi(ll m, int n) {
        if(n==0) return m;
        if(primes[n-1]>=m) return 1;
        if(m<MAXM&&n<MAXN) return dp[n][m];
        return phi(m,n-1)-phi(m/primes[n-1],n-1);
    }
    ll Lehmer(ll m) {
        if(m<MAX) return counter[m];
        ll w,res = 0;
        int i,a,s,c,x,y;
        s=sqrt(0.9+m),y=c=cbrt(0.9+m);
        a=counter[y],res=phi(m,a)+a-1;
        for(i=a;primes[i]<=s;i++) res=res-Lehmer(m/primes[i])+Lehmer(primes[i])-1;
        return res;
    }
}
\end{lstlisting}


\subsection{Int\_128}


\begin{lstlisting}
struct Int_128{
    unsigned long long a,b;
    Int_128(long long x) {
        a=0,b=x;
    }
    friend bool operator < (Int_128 x,Int_128 y) {
        return x.a < y.a || x.a == y.a && x.b < y.b ;
    }
    friend Int_128 operator + (Int_128 x,Int_128 y) {
        Int_128 re(0);
        re.a=x.a+y.a+(x.b+y.b<x.b);
        re.b=x.b+y.b;
        return re;
    }
    friend Int_128 operator - (Int_128 x,Int_128 y) {
        y.a=~y.a;y.b=~y.b;
        return x+y+1;
    }
    void Div2() {
        b>>=1;b|=(a&1ll)<<63;a>>=1;
    }
    friend Int_128 operator * (Int_128 x,Int_128 y) {
        Int_128 re=0;
        while(y.a || y.b) {
            if(y.b&1) re=re+x;
            x=x+x; y.Div2();
        } return re;
    }
    friend Int_128 operator % (Int_128 x,Int_128 y) {
        Int_128 temp=y;
        int cnt=0;
        while(temp<x)
            temp=temp+temp,++cnt;
        for(;cnt>=0;cnt--) {
            if(temp<x)
                x=x-temp;
            temp.Div2();
        } return x;
    }
};
\end{lstlisting}


\subsection{最大割}

\begin{lstlisting}
int mp[N][N], in[N], ou[N], n, ans;
void dfs(int k, int s, int ni, int no) {
    int sum = 0;
    if(s >= ans) return;
    for (int i = 1; i <= ni; i++) sum += mp[in[i]][k];
    if(k == n) { if(s + sum < ans) ans = s + sum; }
    else { in[ni+1] = k; dfs(k+1, s+sum, ni+1, no); }
    sum = 0;
    for (int i = 1; i <= no; i++) sum += mp[ou[i]][k];
    if(k == n) { if(s + sum<ans) ans = s+sum; }
    else { ou[no+1] = k; dfs(k+1, s+sum, ni, no+1); }
}

int main() {
    scanf("%d", &n); ans = 0;
    for (int i = 1; i <= n; i++)
        for (int j = 1; j <= n; j++) {
            scanf("%d", &mp[i][j]);
            ans += mp[i][j];
        }
    int k = ans/2, p=0,q=0;
    for (int i = 1; i < n/2; i++) {
        for (int j = 1; j < n/2; j++) p += mp[i][j];
    } p /= 2;
    for (int i = n/2; i <= n; i++) {
        for (int j = n/2; j <= n; j++) q += mp[i][j];
    } q /= 2;
    ans = p+q; in[1] = 1; dfs(2, 0, 1, 0);
    printf("%d\n", k - ans);
}
\end{lstlisting}


\subsection{等差数列异或和}

\begin{lstlisting}
ll get(ll l, ll p, ll num, ll dis) {
    ll ans = (l/p)*num + (dis/p)*num*(num-1)/2;
    l %= p; dis %= p;
    if(dis*num+l < p) return ans;
    return ans + get((dis*num+l)%p, dis, (dis*num+l)/p, p);
}
`l,r公差为dis;`
ll getSum(ll l, ll r, ll dis) {
    ll num = (r - l) / dis + 1, ans=0, sum;
    for (ll i = 1; i <= r; i<<=1) {
        sum = get(l, i, num, dis);
        if(sum & 1) ans += i;
    } return ans;
}
\end{lstlisting}


\subsection{折半}


礼物折半

\begin{lstlisting}
#include <cstdio>
#include <cstring>
#include <iostream>
#include <algorithm>
#include <vector>
#include <map>

using namespace std;

const int N = 31;
int v[N], w[N];
int s[16][10000], cnt[16];
int n, t;

int slove() {
    memset(cnt, 0, sizeof cnt);
    int l = (n+1)/2, r = n/2;
    for (int i = 0; i < (1<<l); i++) {
        int ret = 0, num = 0;
        for (int j = 0; j < l; j++) {
            if(i&(1<<j)) ret += v[j], num++;
            else ret -= w[j];
        }
        s[num][cnt[num]++] = ret;
    }
    for (int i = 0; i <= l; i++) sort(s[i], s[i]+cnt[i]);
    int ans = 0x3f3f3f3f;
    for (int i = 0; i <(1<<r); i++) {
        int ret = 0, num = 0;
        for (int j = 0; j < r; j++) {
            if(i&(1<<j)) ret -= v[j+l], num++;
            else ret += w[j+l];
        }
        int pos = lower_bound(s[l-num], s[l-num]+cnt[l-num], ret) - s[l-num];
        if(pos >= cnt[l-num] && cnt[l-num] > 0) ans = min(ans, abs(s[l-num][cnt[l-num]-1] - ret));
        else {
            ans = min(ans, abs(s[l-num][pos] - ret));
            if(pos > 0) ans = min(ans, abs(s[l-num][pos-1] - ret));
        }pos = lower_bound(s[r-num], s[r-num]+cnt[r-num], ret) - s[r-num];
        if(pos >= cnt[r-num] && cnt[r-num] > 0) ans = min(ans, abs(s[r-num][cnt[r-num]-1] - ret));
        else {
            ans = min(ans, abs(s[r-num][pos] - ret));
            if(pos > 0) ans = min(ans, abs(s[r-num][pos-1] - ret));
        }
    }
    printf("%d\n", ans);
}

int main() {
    scanf("%d", &t);
    while(t --) {
        scanf("%d", &n);
        for (int i = 0; i < n; i++) scanf("%d", &v[i]);
        for (int i = 0; i < n; i++) scanf("%d", &w[i]);
        slove();
    }
}
\end{lstlisting}


平衡树
\begin{lstlisting}
const int MAXN = 100010;
const int MAXLOG = 20;
const int MAXNODE = MAXN * MAXLOG;
namespace trie{
	int ch[MAXNODE][2], sz[MAXNODE], id = 2;
	inline void ins(int x, int d) {//`插入`
		int i, u = 1;
		for(i = MAXLOG - 1; i >= 0; i--){
			bool v = (x >> i) & 1;
			if(!ch[u][v]) ch[u][v] = newnode();
			u = ch[u][v];
			sz[u] += d;
		}
	}
	inline int nlt(int x) {// `找x的下标`
		int i, u = 1, ret = 0;
		for(i = MAXLOG - 1; i >= 0; i--) {
			bool v = (x >> i) & 1;
			if(v) ret += sz[ ch[u][0] ];
			u = ch[u][v];
			if(!u) break;
		}
		return ret;
	}
	inline int kth(int k) {// `找第K大`
		int i, u = 1, ret = 0;
		for(i = MAXLOG - 1; i >= 0; i--) {
			int lsz = sz[ ch[u][0] ];
			if( k <= lsz ) u = ch[u][0];
			else k -= lsz, ret |= (1 << i), u = ch[u][1];
		}
		return ret;
	}
	int newnode() {//`配合清零`
		ch[id][0] = ch[id][1] = sz[id] = 0;
		return id++;
	}
	void clear(){//`清零`
		ch[1][0] = ch[1][1] = 0;
		id = 2;
	} 
}
\end{lstlisting}