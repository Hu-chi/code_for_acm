\subsection{最大公因数}
找寻a,b的最大公因数, 利用辗转相除法:
\begin{lstlisting}
int gcd(int a, int b) { return b?gcd(b,a%b):a; } // __gcd
\end{lstlisting}
常见公式: 

1. $ gcd(a^n - 1, a^m - 1) = a^{gcd(n, m)} -1$ 

2. 扩展 $a>b, gcd(a, b)=1$ 时 $gcd(a^m-b^m, a^n- b^n) = a^{gcd(m, n)} - b^{gcd(m, n)}$ 

3. $G = gcd(C^1_n, C^2_n, ..., C^{n-1}_n)$ 

* n为质数时G=n 

* n有多个质因子时,G=1 

* n有一个质因子p,G=p 

4. $F_n$ 为斐波那契额数列, $gcd(F_n, F_m) = F_{gcd(n, m)}$

5. 给定两个互素的正整数A,B, 那么他们组合的数$C=p*A+q*B>0$, C最大为AB-A-B,不能组合的个数为$\frac{(A-1)*(B-1)}{2}$ 

6. $(n+1) lcm(C_n^0, C_n^1, .., C^n_n) = lcm(1, 2, 3, .. , n+1)$ 

7. 对于质数p,$(x+y+...+w)^p \%p  == (x^p + y^p + ... + w^p) \% p$ 


\subsection{素数测定}
素数测定常用Miller-Rabin素数测试,是基于费马小定理 $a^p \equiv a(mod \  p )$ p为质数

反过来,如果满足该式子,p大多数为质数

\begin{lstlisting}
#define ll long long
using namespace std;
const int Times = 10;
ll multi(ll a, ll b ,ll m) { ll ans = 0; a%=m;
    while(b) { if(b&1)ans = (ans+a)%m; a = (a+a)%m; b >>= 1; } return ans;
}
ll quick_mod(ll a, ll b, ll m) { ll ans = 1; a %= m;
    while(b) {
        if(b&1)ans = multi(ans, a, m); a = multi(a,a,m); b>>=1;
    } return ans;
}
bool Miller_Rabin(ll n) {
    if(n==2)return true;
    if((n<2) || !(n&1))return false;
    ll m = n-1; int k = 0;
    while((m&1)==0) { k++; m>>=1; }
    for(int i = 0; i < Times; i++) {
        ll a = rand()%(n-1)+1; ll x = quick_mod(a,m,n); ll y = 0;
        for(int j = 0; j < k; j++) {
            y = multi(x,x,n);
            if(y==1 && x!=1 && x!= n-1) return false;
            x = y;
        } if(y!=1) return false;
    } return true;
}
\end{lstlisting}

51Nod 质数检测:

\begin{lstlisting}
import java.util.Scanner;
import java.math.*;
public class Main {
	public static void main(String[] args) {
		Scanner cin = new Scanner(System.in);
		BigInteger x; x=cin.nextBigInteger();
		if(x.isProbablePrime(1)) System.out.println("Yes"); // java has Miller_Rabin
		else System.out.println("No");
	}
}
\end{lstlisting}

\subsection{扩展欧几里得}
对于二元一次方程 ax+by=gcd(a, b) 利用扩展欧几里得可以求得一组可行解。


\begin{lstlisting}
void e_gcd(int a, int b, int &d, int &x, int &y) {
    if(!b) { d=a,x=1,y=0; return; }
    else { e_gcd(b,a%b,d,y,x); y -= (a/b)*x; }
}
\end{lstlisting}
$a>b$时x为正数。可以令y=(y\%a + a) \% a 把y变成正数,再利用原始计算x

\subsection{毕达哥拉斯三元组本原解}
对于$x^2 + y ^2 = z^2$ 我们只考虑本原解,即x,y,z互质。

这样的话,我们不妨假设x为偶数,y,z为奇数,那么就存在互质的数n,m

而且n,m一奇一偶且$m>n$,  有$x=2 * m * n, y  = m^2 - n ^ 2 , z= m^2 + n^2$

{\bfseries POJ-1305}

题意: 给一个正整数n求解,求解n的范围内gcd值为1的勾股数和不是勾股数的个数。

\begin{lstlisting}
#include <bits/stdc++.h>
const int N = 1e6+9;
using namespace std;
int biao[N], n; bool flag[N];
int gcd(int a, int b) { return b?gcd(b,a%b):a; }
void init() {
    for(int i = 2; i*i < N; i++) {
        for(int j = 1; j < i; j++) {
            if(gcd(i,j)!=1) continue; if((i&1)&&(j&1)) continue;
            if(i*i+j*j<N) biao[i*i+j*j]++; // i^2+j^2 = c^2
        }
    } for(int i = 1; i < N; i++) biao[i]+=biao[i-1];
}

int main(){
    init(); while(~scanf("%d",&n)) {
        memset(flag,0,sizeof flag); int ans = 0;
        for(int i = 2; i*i < N; i++) {
            for(int j = 1; j < i; j++) {
                if(gcd(i,j)!=1) continue; if((i&1)&&(j&1)) continue;
                for(int k = 1; k*(i*i+j*j)<=n; k++) // c^2
                    flag[2*i*j*k]=flag[k*(i*i-j*j)]=flag[k*(i*i+j*j)]=1; // used
            }
        } for(int i = 1; i <= n; i++) if(flag[i])ans++;
        printf("%d %d\n",biao[n],n-ans); 
    } return 0;
}
\end{lstlisting}

\subsection{欧拉函数与欧拉定理}

我们定义欧拉函数$\phi (n) $= 小于n且与n互质的数, $\phi(1) = 1, \phi(2) = 1$

1. m与n互素, $\phi(mn)=\phi(n)*\phi(m)$ 

2. $n = p_1^{a_1} p_2^{a_2} p_3^{a_3}... p_n^{a_n}, \phi(n) = n*(1 - \frac{1}{p_1})*(1 - \frac{1}{p_2})*(1 - \frac{1}{p_3})*...*(1 - \frac{1}{p_n}) $

3. $\sum_{d|n} \phi(d) = n$ 即n的因子的欧拉函数之和为n

容斥计算欧拉函数,复杂度$O(\sqrt{n})$:(利用质数可以加快时间)

\begin{lstlisting}
int phi(int n){ int ans = n;
    for(int i = 2; i*i <= n; i++) {
        if(n%i==0) { ans = ans - ans/i; while(n%i==0) n/=i; }
    } if(n>1)ans = ans-ans/n;  return ans;
}
\end{lstlisting}

\subsection{欧拉筛}
欧拉筛是可以做到O(n) 预处理1-n之间所有的质因数, 通过对其修改,我们也能预处理出欧拉函数和莫比乌斯函数以及因数个数。一般欧拉函数不需要修改,可以作为模板.

\begin{lstlisting}
const int N = 5e4+5; // d:number of yinzi
int mu[N],prime[N],d[N],ti[N],phi[N], cnt; bool vis[N];
void init() {
    mu[1] = d[1] = phi[1] = 1; 
    for(int i = 2 ; i < N; i ++) {
        if(!vis[i]) { prime[cnt++] = i; mu[i] = -1; d[i] = 2; ti[i] = 1; phi[i] = i-1; } 
        for(int j = 0; j < cnt; j++) if(i*prime[j] < N) {
            vis[i*prime[j]] = 1;
            if(i%prime[j]==0) {
                ti[i*prime[j]] = ti[i] + 1;
                d[i*prime[j]] = d[i]/(ti[i]+1)*(ti[i]+2);
                phi[i*prime[j]]=phi[i]*prime[j]; break;
            } mu[i*prime[j]] = -mu[i];
            d[i*prime[j]] = d[i]*2; ti[i*prime[j]] = 1;
            phi[i*prime[j]] = phi[i]*(prime[j]-1);
        }
    }
}
\end{lstlisting}
{\bfseries 欧拉筛例题}

题意: 给出一个区间[L,U],求给定区间内的质数距离最小的一对和质数距离最大的一对。$U \le 1e9, U-L \le 1e6$

方法: 预处理$\sqrt{1e9}$  的质数, 然后对[L, U] 区间内去晒(即标记)

\begin{lstlisting}
#define uint unsigned int
const int N = 5e4;
int prime[N], cnt; bool judge[21*N], vis[N];
using namespace std;

void init(){
    for(int i = 2; i<N; i++){ if(!vis[i]) prime[cnt++] = i;
        for (int j = 0; j < cnt; j++) if(i*prime[j] < N) { vis[i*prime[j]] = 1; if(i%prime[j]==0) break; }
    }
}

int main(){ init(); uint u,v;
    while(~scanf("%d%d",&u,&v)) { if(u==1) u=2; // `1不是质数`
        memset(judge,0,sizeof judge);
        for(int i = 0; i < cnt && prime[i] <= v; i++){
            uint j=((u-1)/prime[i]+1)*prime[i]; // `第一个>=u的prime[i]的倍数`
            if(j == prime[i]) j += prime[i]; // `j为质数跳过`
            for(;j <= v; j+=prime[i]) judge[j-u]=1; // `晒去prime[i]的倍数`
        } int st,stlen=N,ed,edlen=0,cot=0,len=0;
        for(int i = 0; i <= v-u; i++) {
            if(!judge[i]) {  // `即u+i为质数`
                if(len && len<stlen) {
                    st = i-len; stlen = len; // `更新最近的质数对`
                } if(len && len>=edlen) {
                    ed = i-len; edlen = len; // `更新最远的质数对`
                } cot++; len = 0; // `统计区间质数个数`
            } len ++ ;
        } if(cot>1)printf("%d,%d are closest, %d,%d are most distant.\n",u+st,u+st+stlen,u+ed,u+ed+edlen);
        else puts("There are no adjacent primes.");
    } return 0;
}	
\end{lstlisting}


{\bfseries 直角三角形个数}

题意: 求$a,b,c<=L$的三角形个数且满足a<b<c.abc互质.$ L < 1e12$

利用三元组本原解枚举n,m实现sqrt(1e12), 容斥去重

\begin{lstlisting}
#include <bits/stdc++.h>
#define ll long long
const int N = 1e6+9;
using namespace std;
int prime[N/10], phi[N], check[40], cnt, num; ll ans, L; bool vis[N];

void init(){
    for (int i = 2; i < N; i++) {
        if(!vis[i]) { prime[cnt++]=i;phi[i]=i-1; }
        for (int j = 0; j < cnt; j++) {
            if(i*prime[j] >= N) break; vis[i*prime[j]] = 1;
            if(i%prime[j]==0) { phi[i*prime[j]]=phi[i]*prime[j];break;}
            phi[i*prime[j]]=phi[i]*(prime[j]-1);
        }
    }
}

void prime_check(int n){ // `质因数分解`
    num = 0;
    if(!vis[n]){ check[num++]=n; return ; } //n `为质数直接返回`
    for(int i = 0; i < cnt && prime[i]*prime[i] <= n; i++) {
        if(n%prime[i]==0){ check[num++]=prime[i]; while(n%prime[i]==0) n/=prime[i]; }
    } if(n>1) check[num++]=n;
}

//`容斥原理计算无关的数fun(j)=j-(j/c1+j/c2+j/c3)+(j/(c1*c2)+j/(c1*c3)+j/(c2*c3))-(j/(c1*c2*c3))。`
void dfs(int k, int r, int s, int n){
    if(k==num){
        if(r&1) ans -= n/s;
        else ans+= n/s;
        return;
    } dfs(k+1, r, s, n);
    dfs(k+1, r+1, s*check[k], n);
}

int main(){
    init(); int t;scanf("%d",&t);
    while(t--){ ans = 0; scanf("%lld",&L);
        int m = (int)sqrt(1.0*L + 0.5);
        for(int i = m; i > 0; i--) { // `枚举m , n<m 只要计算有多少符合n就行`
            int p = (int)sqrt(L - (ll)i*i + 0.5); // `n的上界`
            if ( i<=p ) {
                if(i&1){
                    prime_check(i); dfs(0,0,1,i>>1);
                } else ans += phi[i]; //` 对于偶数,与其互质的数都为奇数可以直接`
            } else {
                // i>p`时,j所取的范围在[1,p]`
                prime_check(i);
                if(i&1)dfs(0,0,1,p>>1);
                else dfs(0,0,1,p);
            }
        } printf("%lld\n",ans);
    }
}
\end{lstlisting}


\subsection{逆元}


对于$am \equiv 1 (mod \ m)$, 这个同余方程中最小的x正整数解叫做a模m的逆元; 也就是说$\frac{1}{a} \equiv x (mod \ m)$

由费马小定理得a对于模数mod为质数的逆元$a^{mod - 2}$可以用快速幂log求得

\begin{lstlisting}
ll qp(ll a, int k) { ll ans = 1; a %= mod;
    while(k) {
        if(k & 1) ans = ans * a % mod;
        a = a*a % mod; k >> = 1;
    } return ans;
}
ll inv(ll a) { return qp(a%mod, mod-2); }
\end{lstlisting}

{\bfseries 预处理逆元:}

我们可预处理1-n的逆元是O(n)的复杂度

\begin{lstlisting}
inv[1]=1;
for (int i=2;i<=n;++i) inv[i]=(mod-mod/i)*inv[mod%i]%mod;
\end{lstlisting}

{\bfseries 预处理n! 的逆元inv}
\begin{lstlisting}
ll fac[N], inv[N];
void init() {
    fac[0] = inv[0] = 1;
    for (int i = 1; i < N; i++) fac[i]=fac[i-1]*i%mod;
    inv[N-1] = qp(fac[N-1], mod-2);
    for (int i = N-2; i; i--) inv[i]=(i+1)*inv[i+1]%mod;
}
\end{lstlisting}

\subsection{等比数列求和}

$S_n = (a + a^2 + .. + a^n) mod M$

如果使用高中的公式那么就是a = 1 时$s_n = n \mod \ M$,  如果a!=1, $S_n = \frac{a - a ^{n+1}}{1 - a}$

这里还有一种方法: 二分

对于n\%2 == 0 $S_n = (1 + a ^{\frac{n}{2}}) S_{\frac{n}{2}}$ ,否则 $S_n = (1 + a^{\frac{n+1}{2}})S_{\frac{n+1}{2}} + a^{\frac{n+1}{2}}$

{\bfseries POJ - 3233 等比矩阵求和}

如果利用上面的递推式子就可以解决矩阵的i次求和问题.

\begin{lstlisting}
using namespace std;
int n,k;
int mod = 1e9+7;
struct Matrix {
    int mp[35][35];
    Matrix() { memset(mp,0,sizeof mp); }
    Matrix(int v){ memset(mp,0,sizeof mp); for(int i = 1; i <= n; i++){ mp[i][i] = v; } }
}a;
inline Matrix multi(Matrix a, Matrix b){
    Matrix c;
    for(int i = 1; i <= n; i++){
        for(int k = 1; k <= n; k++) if(a.mp[i][k])
            for(int j = 1; j <= n; j++) if(b.mp[k][j])
                c.mp[i][j] = (c.mp[i][j] + a.mp[i][k]*b.mp[k][j])%mod;
        }
    } return c;
}
inline Matrix pls(Matrix a, Matrix b){
    for(int i = 1; i <= n; i++) {
        for(int j = 1; j <= n; j++) a.mp[i][j] = (a.mp[i][j]+b.mp[i][j])%mod;
    } return a;
}
inline Matrix powmul(Matrix a, int k){
    Matrix c = Matrix(1);
    while(k){
        if(k&1)c = multi(c,a);
        a = multi(a,a); k>>=1;
    } return c;
}
inline Matrix S(Matrix a, int k){
    if(k==1) return a;
    Matrix temp = Matrix(1);
    if(k&1){
        Matrix temp2 = powmul(a,(k+1)/2);
        return pls(multi(pls(temp, temp2), S(a,(k-1)/2)), temp2);
    } else {
        Matrix temp2 = powmul(a,k/2);
        return multi(pls(temp, temp2), S(a,k/2));
    }
}

int main(){
    while(~scanf("%d%d%d",&n,&k,&mod)) {
        for(int i = 1; i <= n; i++) {
            for(int j = 1; j <= n; j++) 
                scanf("%d",&a.mp[i][j]);
        } a = S(a,k);
        for(int i = 1 ; i <= n; i++) {
            for(int j = 1; j < n; j++) printf("%d ",a.mp[i][j]);
            printf("%d\n", a.mp[i][n]);
        }
    } return 0;
}
\end{lstlisting}

这个复杂度时有点高的,还可以构造矩阵利用矩阵快速幂一步到位,但是模数不为质数就得用这个了

我们要求的矩阵和为$S_n$

那么 $$ \begin{bmatrix} S_n \\ A^n  \end{bmatrix}  =  \begin{bmatrix}1 & A \\ 0  & A  \end{bmatrix} \begin{bmatrix} S_{n-1} \\ A^{n-1}  \end{bmatrix} $$ 

\subsection{组合数计算}

从n个物品中选取k个物品有$C_n^{k}$ 种方法。

组合数表预处理 mod任意:

\begin{lstlisting}
void init(){
    c[0][0] = 1;
    for (int i = 1; i < N; i++) {
        c[i][0] = c[i][i] = 1;
        for (int j = 1; j < i; j++) c[i][j] = (c[i-1][j] + c[i-1][j-1]) % mod;
    }
}
\end{lstlisting}

预处理阶乘, mod只能为质数

\begin{lstlisting}
void init() {
    fac[0] = inv[0] = 1;
    for (int i = 1; i < N; i++) fac[i]=fac[i-1]*i%mod;
    inv[N-1] = qp(fac[N-1], mod-2);
    for (int i = N-2; i; i--) inv[i]=(i+1)*inv[i+1]%mod;
}
ll C(int n, int m) { return fac[n]*inv[m]%mod*inv[n-m]%mod; }
\end{lstlisting}

Lucas 模数必须为小质数

\begin{lstlisting}
int Lucas(int n, int m){
    if(m==0)return 1;
    return C(n%p,m%p)*Lucas(n/p,m/p)%p;
}
\end{lstlisting}


\subsection{中国剩余定理CRT}


有n组同余方程
$$
x \equiv a_1 (mod \ m_1) \\ x \equiv a_2 (mod \ m_2) \\ x \equiv a_3 (mod \ m_3) \\ ... \\ x \equiv a_n (mod \ m_n) \\ 
$$
利用CRT既可以解模数互质的CRT
\begin{lstlisting}
void e_gcd(int a, int b, int &d, int &x, int &y) {
    if(!b){ d = a; x = 1; y = 0; return ; }
    e_gcd(b, a%b, d, y, x); y -= x*(a/b);
}
//`中国剩余定理`
int CRT(int a[], int m[], int n){
    int M = 1; int ans = 0;
    for(int i = 0; i < n; i++) M *= m[i];
    for(int i = 0; i < n; i++) {
        int x,y,d; int Mi = M/m[i];
        e_gcd(Mi, m[i], d, x, y);
        ans = (ans + Mi*x*a[i])%M;
    } if(ans<0)ans+= M;
    return ans;
}
\end{lstlisting}
对于普通中国剩余定理要求的$m_1,m_2,…,m_k$互素.但如果发生不互素时,需要采用两两合并. 

\subsection{反素数}

f(n)为n的因数个数,对于$0<i<n,f(i) < f(n)$, 则称n为反素数。

反素数一般满足质因数的幂次随着质因数的大小增大而减小 ,即若$ n = 2^{t_1}3^{t_2},$ 则必有$t_1 \ge t_2$

{\bfseries 求最小的数使其因数为n,CF27E}

$n<=1000$,我们贪心的认为其质因数不会超过前50个质数,而且幂次不会超过63。

\begin{lstlisting}
void dfs(int depth, ll tmp, int num) {  //depth `为第几个质数, tmp为当前的值,um为因数个数`
    if(num > n || depth>=cnt) return ;
    if(num == n && ans > tmp) { ans = tmp; return; // `满足的情况下更新` } 
    for(int i= 1; i <= 63; i++){ // `枚举幂次, 事实上可以记录上一次的幂次来优化剪枝`
        if(ans/prime[depth]<tmp || num*(i+1)>n) return; // `用除法防止溢出`
        dfs(depth+1, tmp*=prime[depth], num*(i+1));
    }
}
\end{lstlisting}


{\bfseries 求1-n因数个数最大的数 URAL 1748}


\begin{lstlisting}
void dfs(int depth, ll ansa, int ansb, int limit){ //`ansa为值, ansb为因数个数`
    if(depth >= tot || ansa > n)return;
    if(ansb > b|| ((ansb == b) && a>ansa)){
        a = ansa; b = ansb;
    } for(ll i = 1; depth < tot && i <= limit; i++){
        if(ansa > n/prime[depth])break;
        dfs(depth+1, ansa*=prime[depth], ansb*(i+1), i);
    } return ;
}
\end{lstlisting}

{\bfseries HDU4542}

题意: X mod ai = bi  对于ai在区间 [1, X] 范围内每个值取一次时,有K个ai使bi等于0 。 告诉你有K个ai使得(不使得) bi等于0, 找最小的X

解析:K个相等的时候我们去dfs暴力枚举,K个不相等的话这个数一定不大,先预处理

\begin{lstlisting}
void init() {
    for(int i = 1; i < N; i++) d[i] = i; 
    for(int i = 1; i < N; i++) {
        for(int j = i; j <N; j+=i) d[j]--; // `容斥减去因子个数`
        if(!d[d[i]]) d[d[i]] = i; // `更新因子个数为d[i]位置的数为i, 节约了空间,这样也保是最小的`
        d[i] = 0
    }
}
void dfs(int depth, ll ansa, int ansb, int limit){
    if(ansb > k || depth > 16) return ;
    if(ansb == k && ansa < ans) { ans = ansa; return ; } 
    for(int i = 1; i <= limit; i++) {
        if(ansa > pos/prime[depth] || ansb*(i+1)>k) break;
        ansa*=prime[depth];
        if(k%(ansb*(i+1))==0) dfs(depth+1, ansa, ansb*(i+1), i);
    }
}
int main(){
    init(); int T,t=0; scanf("%d",&T);
    while(t++<T) {
        scanf("%d%d",&type,&k); ans = flag = 0;
        if(type) {
            if(d[k]) printf("Case %d: %d\n",t,d[k]);
            else printf("Case %d: Illegal\n",t);
        } else {
            ans = ~0ull; dfs(0,1,1,63);
            if(ans > inf) printf("Case %d: INF\n",t);
            else printf("Case %d: %llu\n",t,ans);
        }
    } return 0;
}

\end{lstlisting}

\subsection{指数循环节}

$$a^b \equiv a^{b \% \phi(c)+\phi(c)}(mod \ c),b>\phi(c) $$

{\bfseries HDU4335}

这里求$n^{n!} \equiv b(mod \ p)$有多少个成立的n,对于指数\%phi(p)为0后的数,是存在循环节的。只要记录一下一次循环的个数p,最后多余的部分去跑一下记忆化的就知道有多少组了。

\begin{lstlisting}
#include <bits/stdc++.h>
#define ll unsigned long long
const int maxn = 1e5+5;
using namespace std;

ll phi(ll n) { ll res = n;
    for(ll i = 2; i <= sqrt(n); i++) {
        if(n%i==0) { res = res - res/i; while(n%i==0) n/=i; }
    } if(n > 1) res = res - res/n; return res;
}

ll quickmod(ll a, ll b, ll c){ ll ans = 1;
    while(b) {
        if(b&1) ans = ans*a%c;
        a = a*a%c; b >>= 1;
    } return ans;
}

ll b, p, m, fa[maxn];
int main(){
    int t=0,T;scanf("%d",&T);
    while(t++<T){
        scanf("%I64u%I64u%I64u", &b, &p, &m); printf("Case #%d: ",t);
        if(p==1){
            if(m==18446744073709551615ull) printf("18446744073709551616\n"); // `防止溢出`
            else printf("%I64u\n",m+1); continue;
        }else{
            ll i = 0, ans = 0, fac = 1; ll phic = phi(p);
            for(i = 0; i <= m && fac<=phic; i++) {
                if(quickmod(i, fac, p) == b) ans ++; // `计算n! 小于 phi(p)的个数`
                fac *= i+1;
            } fac = fac%phic; //`指数循环节公式`
            for(;i <= m && fac; i++){
                if(quickmod(i, fac + phic, p) == b) ans++;
                fac = (fac*(i+1))%phic;
            } // `至多跑一个循环或者不超过M`
            if(i <= m){
                ll cnt = 0;
                for(int j = 0; j < p; j++){
                    fa[j] = quickmod(i+j, phic, p); // `记忆化p长度的循环`
                    if(fa[j] == b) cnt ++;
                }
                ll idx = (m-i+1)/p; // `循环个数`
                ans += cnt * idx;   // `循环的种数`
                ll remain = (m-i+1)%p; // `不足一个循环的数`
                for(int j = 0; j < remain; j++) if(fa[j] == b) ans ++ ;
            } printf("%I64u\n",ans);
        }
    } return 0;
}
\end{lstlisting}

{\bfseries 牛客练习赛 22E}

题意:给一个长为n的序列,m次操作,每次操作: 

1.区间[l, r]加x

2.对于区间[l,r],查询$a[l]^{a[+1]^{a[l+2]...}} mod p$,一直到$a_r$, 请注意每次的模数不同。

区间加值我们可以考虑使用线段树lazy标记,查询就直接使用指数循环节公式倒过来计算就行了。

这道题因为数据太多,用文件读入才能过

\begin{lstlisting}
#include <bits/stdc++.h>
using namespace std;
using ll = long long;
const int N = 5e5+9;
const int M = 2e7+9;
ll val[N]; bool judge[M];
int prime[M],p[M],mod[N],tot;
using namespace fastIO;

inline void init(){ p[1] = 1;
    for(int i = 2; i < M; i++) {
        if(!judge[i]) { prime[tot++]=i; p[i] = i-1; }
        for(int j = 0; j < tot; j++) {
            if(i*prime[j]>=M)break; judge[i*prime[j]] = 1;
            if(i%prime[j] == 0) { p[i*prime[j]] = p[i]*prime[j]; break;} 
            p[i*prime[j]]=p[i]*(prime[j]-1);
        }
    }
}
inline int phi(int n) { return p[n]; }
inline int quickmod(ll a, int b, int c) {
    ll ans = 1; a %= c;
    while(b) {
        if(b&1) ans = ans*a%c;
        a = a*a%c; b >>= 1;
    } return ans;
}

inline int slove(ll a, int b, int c) { // `指数循环节公式`
    if(b*log(a) >= log(c)) return quickmod(a, b, c)+c;
    else return quickmod(a, b, c);
}
int n,m,op;

// `线段树模板`
namespace sgt {
    long long chg[1000010];
    int son[1000010][2],cnt,root;
    void build(int &x,int l,int r) {
        x=++cnt;
        if(l==r) return ;
        int mid=(l+r)>>1;
        build(son[x][0],l,mid);
        build(son[x][1],mid+1,r);
    } void update(int a,int b,int k,int l,int r,long long v) {
        if(a>b || l>r)return;
        if(a<=l && b>=r)chg[k]+=v;
        else {
            int mid=(l+r)>>1;
            if(b<=mid)update(a,b,son[k][0],l,mid,v);
            else if(a>mid)update(a,b,son[k][1],mid+1,r,v);
            else update(a,mid,son[k][0],l,mid,v),update(mid+1,b,son[k][1],mid+1,r,v);
        }
    } long long query(int a,int k,int l,int r,long long v) {
        if(l==r)return v+chg[k]+val[a];
        int mid=(l+r)>>1;
        if(a<=mid)return query(a,son[k][0],l,mid,v+chg[k]);
        else return query(a,son[k][1],mid+1,r,v+chg[k]);
    }
}

int main(){ 
    init(); read(n); read(m);
    for (int i = 1; i <= n; i++) read(val[i]);
    sgt::build(sgt::root,1,n);
    for (int t = 0; t < m; t++) {
        read(op); ll x; int l, r;
        if(op==1){
            read(l); read(r); read(x);
            sgt::update(l, r, sgt::root, 1, n, x); // `区间加值`
        } else {
            read(l); read(r); read(x); int ans = 1; mod[l] = x;
            for(int i = l+1; i <= r; i++){
                mod[i] = phi(mod[i-1]); // `预处理幂次的幂次的模数`
                if(mod[i] <= 2) r = i; // `某幂次<2那么基本就是0了可以停止了`
            } for(int i = r; i >= l; i--){
                ans = slove(sgt::query(i, sgt::root, 1, n, 0), ans, mod[i]); //`单点查询,倒着计算`
            } printf("%d\n",ans%x);
        }
    } return 0;
}
\end{lstlisting}


\subsection{积性函数}

{\bfseries 积性函数的定义}

1. 若f(n)的定义域为正整数域,值域为复数,即$f:Z^+ - Cf:Z^+ - C$,则称f(n)为**数论函数**。 

2. 若f(n)为数论函数,且f(1)=1,对于互质的正整数p,q有f(p⋅q)=f(p)⋅f(q),则称其为**积性函数**。
 
3. 若f(n)为积性函数,且对于任意正整数p,q都有f(p⋅q)=f(p)⋅f(q),则称其为**完全积性函数**。

{\bfseries  常见的积性函数}

1. 除数函数$\sigma_k(n) = \sum_{d|n} d^k$, 表示n的约数的k次幂和

2. 约数个数函数$d(n) = \sigma_0(n) = \sum_{d|n} 1$ , 表示约数个数

3. 约数和函数$\sum_{d|n} d$

4. 欧拉函数$\phi(n)=\sum_{i=1}^{n} [(n, i) == 1]$, 对于n>=2 有 $\phi(n)=\sum_{i=1}^{n} [(n, i) == 1] · i = n\phi(n)/2 $

5. 莫比乌斯函数$u(n)$, 与恒等函数互为逆元

6. 元函数 $e(n) = [n == 1]$

7. 恒等函数$I(n) = 1$

8. 单位函数$id(n) =n$

9. 幂函数$id^k(n) = n^k$

{\bfseries 两个常用的公式}

1. $e(n) = \sum_{d|n}u(d)$, 可以将u(d)看作是容斥的系数

2. $n = \sum_{d|n} \phi(d)$

{\bfseries 积性函数性质}

若f(n)为积性函数,对于正整数$n = p_{1}^{k_1}p_{2}^{k_2}...p_{n}^{k_n}$

有$f(n)=\prod_{i=1}^{t}{f(p_i^{k_i})}  $, 对于完全积性函数有$f(n)=\prod_{i=1}^{t}{f(p_i)^{k_i}}$


\subsubsection{分块技巧}

题意:有一份包含个 bug 的n( $1 \le n \le 10^6 $)行代码,运一次到崩溃需要的时间为 r($ \le 1 r \le 10^9 $)。你可以任意行添加 printf 语句来输出调试,即你知道是否在执行 printf 语句前就崩溃了。每设置一个 printf语句需要花费 p( $1 \le p \le 10^9$)
问最坏的情况下,最少需要多少时间可以定位bug的行数

解释:我们设置f(n)为n行代码debug需要的最少时间。那么对于f(n)我们时可以由前面的状态转移过来的,我们对于n行代码分成2块,那么我们就需要f ( (n+1)/2  ) + r  + p的时间, 我们分成k块,那么最大的块的大小就为 (n+k-1)/k 向下取整个, 所以总体时间就是 f((n+k-1)/k) + r + (k-1)*p;

这样我们从1-n递推过来就是O($n^2$) 的算法, 我们要考虑优化, 随着k增大,我们会发现有很多(n+k-1)/k 是相同,但是我们应该取相同种k最小的,所以我们反过来枚举(n+k-1)/k的大小,也就是每个块最大为l, 那么最多有(n+l-1)/l 块, 所以这样总体时间就是f(l) + r +( (n+l-1)/l )* p;

对于n $\sqrt n$的算法还是很吃力的,我们考虑使用记忆化搜索来降低复杂度。

\begin{lstlisting}
#include <bits/stdc++.h>
using ll = long long;
using namespace std;
const int N = 1e6+9;
int n, r, p, vis[N], t;
ll f[N];
ll dfs(int n){
    if(vis[n] == t || n == 1) return f[n];  // `如果已经计算过直接返回`
    ll ans = 1e18+7; vis[n] = t; // `表示t组样例的时候f[n]被跟新了`
    for(int j = 1; j <= sqrt(n+0.5); j++) {
        ans = min(ans, dfs((n+j)/(j+1)) + r + 1LL*j*p); // `枚举j+1块`
        ans = min(ans, dfs(j)+r+((n+j-1)/j-1LL)*p);     // `枚举每块最大为j个`
    } return f[n] = ans; // `记忆化`
}

int main(){
    while(~scanf("%d%d%d", &n, &r, &p)) {
        t++; printf("%lld\n", dfs(n));   // `每组,记忆话搜索一下`
    } return 0;
}
\end{lstlisting}

\subsection{迪利克雷卷积}

设f,g为两个数论函数,则满足$h(n) = \sum_{d|n} f(d)g(\frac{n}{d})$ 称为f与g的迪利克雷卷积。

{\bfseries 积性函数例题}

题意:f(n)=为[0, n)种选2个数a,b且ab不为n的倍数的方案数。 求$g(n) = \sum_{d|n} f(d) \% 2^{64}$
,$1 \le T \le 2e4, 1 \le n \le 1e9$

解析:我们先考虑f(n) , f(n) = n*n - ab为n的倍数的方案数,即$f(n) = n^2 -  \sum_{d|n}\phi(\frac{n}{d})d$

 令$h(n) = \sum_{d|n}\phi(\frac{n}{d})d$

 那么原式=$\sum_{d|n}d^2 - \sum_{d|n}h(d)$, 很显然这两个函数都是积性函数,我们如果直接对其质因数分解是可以做的。不过处理后面的还是很麻烦。

 $\sum_{d|n}\sum_{w|d}\phi(w) * \frac{d}{w} = \sum_{w|n}w· \sum_{\frac{d}{w} |\frac{n}{w} } \phi(\frac{d}{w}) = \sum_{w|n} n = n·d(n)$

 这样处理就方便了很多


\begin{lstlisting}
#include <bits/stdc++.h>
const int N = 1e5+8;
using namespace std;
using ll = unsigned long long;
using lll = unsigned __int128;
ll a[N], num[N], cnt, prime[N], tot, n;
bool vis[N];
void init(){ 
    for(int i = 2; i < N; i++) {
        if(!vis[i]) prime[tot++] = i;
        for (int j = 0; j < tot; j++) {
            if(i*prime[j]>=N) break;
            vis[i*prime[j]] = 1;
            if(i%prime[j]==0) break;
        }
    }
}
void d(int n){
    #define j prime[i]
    for(int i = 0; 1LL*j*j <= n && i < tot; i++){
        if(n%j==0) {
            num[cnt] = 0; a[cnt] = j;
            while(n%j==0) { num[cnt]++; n /= j; } cnt++;
        }
    } if(n>1) a[cnt]=n,num[cnt++]=1;
    #undef j
}
lll qp(lll a, int k){
    lll ans = 1;
    while(k) {
        if(k&1) ans *= a;
        a *= a; k >>= 1;
    } return ans;
}
ll p(ll n){
    if(n == 1) return 1; ll ans = 1;
    for (int i = 0; i < cnt; i++) ans *= (qp(a[i], num[i]*2+2)-1)/(a[i]*a[i]-1);
    return ans;
}
ll q(ll n) {
    ll ans = n;
    for (int i = 0; i < cnt; i++) ans *= num[i] + 1;
    return ans;
}
int main(){
    int t;scanf("%d", &t); init();
    while(t--) {
        scanf("%lld", &n); cnt = 0; d(n);
        printf("%llu\n", p(n) - q(n));
    } return 0;
}

\end{lstlisting}

{\bfseries 积性函数习题 757E}

题意: $f_0 (n)$为 (p, q) == 1, 且p*q = n的个数, $f_{r+1}(n) = \sum_{u * v=n} \frac{f_{r}(u) + f_{r}(v)}{2} $,求$f_r(n) \%1e9+7$

解释: 显然$f_0(n) = 2^{d(n)}$,d(n)为质因数个数,这是积性函数,我们可以对其化简$f_0(n) = f_0(p_1^{k_1})f_0(p_2^{k_2})...f_0(p_t^{k_t}) = \prod_{i=1}^t(1+k_i)$

$f(p^k)$只与k有关,而k又是logn的,由$f_r(n)$定义可知这是一个积性函数,所以做r次前缀和就可以了。


\begin{lstlisting}
#include <bits/stdc++.h>

using namespace std;
using ll = long long;
const int N = 1e6+9;
ll f[N][21];
const int mod = 1e9+7;
int prime[N], cnt; bool vis[N];
int q, r, n;

void init(){
    for (int i = 2; i < N; i++) {
        if(!vis[i]) prime[cnt++] = i;
        for (int j = 0; j < cnt; j++) {
            if(i*prime[j] >= N) break; vis[i*prime[j]] = 1;
            if(i%prime[j]==0) break;
        }
    } f[0][0] = 1;
    for(int i = 1; i < 21; i++) f[0][i] = 2;
    for(int i = 1; i < N; i++) {
        f[i][0] = 1;
        for (int j = 1; j < 21; j++) f[i][j] = (f[i][j-1]+f[i-1][j])%mod; //`前缀和`
    }
}

ll d(int n){ 
    #define j prime[i]
    ll ans = 1;
    for (int i = 0; i < cnt && j*j <= n; i++) {
        if(n%j==0) {
            int t = 0;
            while(n%j==0)n/=j,t++; // `算$p^k$`
            ans = (ans * f[r][t])%mod;  // `计算每个质因数的贡献`
        }
    } if(n>1) ans = (ans * f[r][1]) % mod; 
    return ans;
    #undef j
}

int main() {
    init(); scanf("%d", &q);
    while(q--) {
        scanf("%d%d", &r, &n);
        printf("%lld\n", d(n));
    } return 0;
}
\end{lstlisting}

\subsection{莫比乌斯反演}

$$
f(n) = \sum_{d|n} g(d)  <=> g(n) =  \sum_{d|n}u(d)f(\frac{n}{d})
$$

这就是莫比乌斯函数,对于已知的f(n),我们都可以利用该等式来反过来求g(n);

{\bfseries 求互质的数对数}

题意:求$1\le i\le b, 1\le j\le d$ 由多少对(i,j)满足gcd(i, j) = k

解析:我们先解决$1\le i\le n, 1\le j\le m. gcd(i, j)== 1$的问题

令F(d) 为有多少对i,j满足gcd(i, j) == d的倍数

f(d) 为有多少对gcd(i, j) == d, 那么我们就有$F(d) = \sum _{}f(id)$, $f(1) = \mu(1)F(1) + \mu(2)F(2)+...$

显然$f(1) = \sum_{i=1}^{min(n, m)} \mu(i) F(i)= \sum_{i=1}^{min(n, m)} \mu(i) \frac{n}{i} \frac{m}{i}$, 

利用分块就可以做到$\sqrt{n}+\sqrt{m}$ 查询

所以对于原问题,b,d/=k就转成了gcd(i,j)==1, 再去减去不符合的就可以了

\begin{lstlisting}
#include <bits/stdc++.h>
#define ll long long
const int N = 1e5+10;
using namespace std;
bool vis[N];
int mu[N], prime[N], cnt;
inline void init(){ 
    mu[1] = 1;
    for(int i = 2; i < N; i++){
        if(!vis[i]){ prime[cnt++] = i; mu[i] = -1; }
        for(int j = 0; j < cnt; j++){
            if(i*prime[j]>N)break;
            vis[i*prime[j]]= 1;
            if(i%prime[j] == 0){ mu[i*prime[j]] = 0; break; }
            mu[i*prime[j]] = -mu[i];
        }
    } for (int i = 2; i < N; i++) mu[i] += mu[i-1];
}
inline ll slove(int l, int r){
    ll ans = 0;
    for(int i = 1, tmp; i <= l; i=tmp+1) {
        tmp = min(l/(l/i), r/(r/i));
        ans += (ll)(mu[tmp]-mu[i-1])*(l/i)*(r/i);
    } return ans;
}
int main(){
    int b,d,k,T=0, t; scanf("%d",&t); init();
    while(t--) {
        scanf("%*d%d%*d%d%d",&b,&d,&k);
        if(!k) {
            printf("Case %d: 0\n",++T); continue;
        } b/=k;d/=k; if(b > d) swap(b, d);
        printf("Case %d: %lld\n",++T,slove(b,d) - slove(b, b)/2);
    } return 0;
}
\end{lstlisting}

{\bfseries GCD表中的质数}

题意:有一个M * N的表格,行与列分别是1 - M和1 - N,格子中间写着行与列的最大公约数Gcd(i, j)($1 \le i \le M, 1 \le j \le N)$。  求质数个数。

解析:我们枚举p,利用上面的结论就有$ans = \sum_{p}\sum_{i=1}^{\frac{n}{p}} \lfloor \frac{n}{ip} \rfloor \lfloor \frac{m}{ip} \rfloor$ , 根据这个公式的复杂度$O(T (\frac{n}{log\ n}·(\sqrt{n}+\sqrt{m}) ) )$ , 这样复杂度有点高,我们对原始做一些优化,交换两个求和符号就可以得到$\sum_{i=1}^{n}\lfloor \frac{n}{i} \rfloor \lfloor \frac{m}{i} \rfloor \sum_{j|i}\mu(\frac{i}{j}) \&\& $ j为质数

如果我们预处后面的一个求和,那么就可以做到分块查询

\begin{lstlisting}
#include <bits/stdc++.h>
using namespace std;
const int N = 5e6+9;
using ll = long long;
int prime[N/5], mu[N], cnt, n, m;
ll f[N]; // `为后一个求和`
bool vis[N];

inline void init(){
    mu[1] = 1;
    for (int i = 2; i < N; i++) {
        if(!vis[i]) {prime[cnt++] = i; mu[i]=-1; f[i]=1;} // `对于质数f[i] = mu[1] = 1;`
        for (int j = 0; j < cnt; j++) {
            if(i*prime[j]>=N) break;
            vis[i*prime[j]] = 1;
            if(i%prime[j]==0) { f[i*prime[j]] = mu[i]; break; } //`prime[j]次数为2时,之前的u都变成了0,只剩下p=prime[j],所以f[i*prime[j]]=mu[i];`
            f[i*prime[j]] = -f[i] + mu[i]; mu[i*prime[j]] = -mu[i]; //`次数为1时,u变成原来的相反数,再多上新增的。`
        }
    } for (int i = 2; i < N; i++) f[i] += f[i-1]; // `前缀和`
}

inline ll solve(int n, int m) { // `分块查询`
    ll ans = 0;
    for (int i = 1, j; i <= n; i=j+1) {
        j = min(n/(n/i), m/(m/i));
        ans += (f[j]-f[i-1])*(n/i)*(m/i);
    } return ans;
}

int main(){
    int T; scanf("%d", &T); init();
    while(T--){
        scanf("%d%d", &n, &m); if(n > m) swap(n, m);
        printf("%lld\n", solve(n, m));
    }
}
\end{lstlisting}

诸如此类的还有

1. $\sum_{a=1}^{N} \sum_{b=1}^M gcd(a, b) = \sum_{d=1}^{min(N, M)} \phi(d) \lfloor \frac{N}{d} \rfloor \lfloor \frac{M}{d} \rfloor$

2. $\sum_{a=1}^{N} \sum_{b=1}^M lcm(a, b) = \frac{1}{4}\sum_{d=1}^{min(N, M)} \lfloor \frac{N}{d} \rfloor \lfloor \frac{M}{d} \rfloor(\lfloor \frac{N}{d} \rfloor + 1)(\lfloor \frac{M}{d} \rfloor+1) d \sum_{d'|d}d'u(d') $

3. $\sum_{a=1}^{N} \sum_{b=1}^N lcm(a, b) = \sum_{i=1}^N (-i+2\sum_{j=1}^ilcm(i,j))$, 预处理后可以O(1)输出

4. $\sum_{i=1}^n gcd(i, n) = \sum_{d|n}\phi(d) \frac{n}{d}$

5. $\sum_{i=1}^n e(gcd(i, n)) = \sum_{d|n}\mu(d) \frac{n}{d}$

6. 约束和之和$\sum_{i=1}^n = \sum_{i=1}^n i \lfloor \frac{n}{i} \rfloor$

7. $g(n)=\frac{n}{2}(\phi(n)+e(n))$

8. $\sum_{i=1}^n lcm(i, n) = n\sum_{d|n}g(\frac{n}{d})$

9. $\sum_{i=1}^n\sum_{j=1}^mij = \frac{n(n+1)m(m+1)}{4}$

10. $\sum_{i=1}^n\sum_{j=1}^mij * e(gcd(i,j)) = \sum_{d=1}^n \mu(d) \sum_{i=1}^{\lfloor \frac{n}{d}\rfloor}\sum_{j=1}^{\lfloor \frac{m}{d}\rfloor}ij$

\subsection{杜教筛}

设$S(n)  = \sum_{i=1}^n f(i)$, 若存在数论函数g,设$h=f*g$, 则有$\sum_{i=1}^{n}h(i)=\sum_{i=1}^ng(i)S(\lfloor \frac{n}{i} \rfloor)$ , 化简得

$g(1)S(n)=\sum_{i=1}^nh(i)-\sum_{i=2}^ng(i)S(\lfloor \frac{n}{i} \rfloor)$, 

如果我们可以$O(\sqrt{n})$计算$\sum_{i=1}^n h(i)​$, O(1)计算g(n)前缀和,那么我们就可以分块查询S(n), 复杂度为$O(n^{\frac{3}{4}})​$, 如果预处理前$n^{\frac{2}{3}}$项就可以优化到$O(n^{\frac{2}{3}})$.

{\bfseries 欧拉函数前缀和 51nod 1239}

题意:求$\sum_{i=1}^n \phi(n)$ , $n < 1e11;$

解析:$S(n)=\sum_{i=1}^n i - \sum_{i=2}^n 1 *S(\lfloor \frac{n}{i} \rfloor)$

\begin{lstlisting}
#include <bits/stdc++.h>
using namespace std;
using ll = long long;
const int N = 1.5e7+9;
const int mod = 1e9+7;
int prime[N], phi[N], cnt;
bool vis[N];
unordered_map<ll, ll> mp; // `记忆化`
void init(){ 
    phi[1] = 1;
    for (int i = 2; i < N; i++) {
        if(!vis[i]) {
            prime[cnt++] = i; phi[i] = i - 1;
        } for (int j = 0; j < cnt; j++) {
            if(i*prime[j] >= N) break;
            vis[i*prime[j]] = 1;
            if(i%prime[j] == 0) { phi[i*prime[j]] = phi[i]*prime[j]; break; }
            phi[i*prime[j]] = phi[i]*(prime[j]-1);
        }
    } for (int i = 2; i < N; i++) phi[i] = (phi[i-1] + phi[i])%mod;
}
ll qp(ll a, int k){
    ll ans = 1;
    while(k) {
        if(k & 1) ans = ans * a % mod;
        a = a* a %mod; k >>= 1;
    } return ans;
}ll inv2 = qp(2, mod-2);

ll dfs(ll n) {
    if(n < N) return phi[n]; // `预处理`
    if(mp.find(n) != mp.end()) return mp[n];
    ll ans = n%mod*(n%mod+1)%mod*inv2%mod; // n*(n+1)/2
    for (ll i = 2,j; i <= n; i=j+1) { // `分块`
        j = n/(n/i);
        ans = (ans - (j-i+1)%mod*dfs(n/i)%mod) %mod; // 1 `从i加到j = (j-i+1)`
    } return mp[n]=(ans%mod + mod)%mod;
}

ll n;
int main() {
    init(); scanf("%lld", &n);
    return printf("%lld\n", dfs(n)), 0;
}

\end{lstlisting}

{\bfseries 莫比乌斯函数前缀和}

题意:求$\sum_{i=1}^n \mu(i)$, $i <= 1e10$

解析:u*I=e, 所以$S(n) = 1 - \sum_{i=2}^n S(\lfloor \frac{n}{i} \rfloor )$

\begin{lstlisting}
#include <bits/stdc++.h>
using namespace std;
using ll = long long;
ll a, b;
const int N = 1e7+9;
int prime[N], cnt;ll mu[N];
bool vis[N];
unordered_map<ll, ll> mp;
void init(){ // `欧拉筛`
    mu[1] = 1;
    for (int i = 2; i < N; i++) {
        if(!vis[i]) { prime[cnt++] = i; mu[i] = -1; }
        for (int j = 0; j < cnt; j++) {
            if(i*prime[j] >= N) break; vis[i*prime[j]] = 1;
            if(i%prime[j]==0) break; mu[i*prime[j]] = - mu[i];
        }
    } for (int i = 2; i < N; i++) mu[i] += mu[i-1];
}
ll dfs(ll n){
    if(n < N) return mu[n];
    if(mp.count(n)) return mp[n];
    ll ans = 1; // e
    for (ll i = 2,j; i <= n; i=j+1){
        j = n/(n/i); ans -= dfs(n/i)*(j-i+1); // `和上一题一样`
    } return mp[n]=ans;
}

int main() {
    scanf("%lld%lld", &a, &b); init();
    return printf("%lld\n", dfs(b)-dfs(a-1)), 0;
}
\end{lstlisting}

一道习题 :

$N^2-3N+2=\sum_{d|N} f(d)$, 求$\sum_{i=1}^Nf(i)\ mod \ 1e9+7, N \le 1e9$

这道题很容易用杜教筛做,不过他的前1e7前缀预处理可以用容斥来写

\begin{lstlisting}
inline void init(){
    for(int i = 1; i < N; i++){
        f[i] = (i-1LL)*(i-2LL)%mod;
    } for (int i = 1; i < N; i++) {
        for (int j = i+i; j < N; j+=i)
            f[j] = ((f[j]-f[i])+mod)%mod; // `容斥`
    } for (int i = 2; i < N; i++) f[i]=(f[i]+f[i-1])%mod; // `前缀和`
}
\end{lstlisting}

\subsection{Extended Eratosthenes Sieve**洲阁筛** }

洲阁筛时一种在$O(\frac{n^{\frac{3}{4}}}{log \ n})$ 时间内计算大多数积性函数的前缀和的方法。

F(x)是一个积性函数,要求在低于线性的时间内求出。当p为质数时,$F(p^c)$是关于p的低阶多项式。

对于1-n中的所有数,我们按照是否含有大于$\sqrt{n}$ 的质因子分为两类,则显然有
$$
\sum_{i=1}^n F(i) = \sum_{i=1 \&\& i \ has't \ bigger \ than \sqrt{n} prime \ d}^n F(i)(1 + \sum_{\sqrt{n}\le j \le \lfloor \frac{n}{i} \rfloor \& \& j\ is \ prime} F(j) )
$$

事实上,我们可以预处理 $ 1 \le i < \sqrt{n} $ 的F, 那么现在的问题就是要解决

1. $\sum_{\sqrt{n} < j < \le \lfloor \frac{n}{i} \&\&j is prime\rfloor} F(j)$

我们令g(i, j) 表示[1, j]中与前i各质数互质的数的k次幂和。显然有

$$
g(i, j) = g(i-1, j)-p^k_ig(i-1, \lfloor \frac{j}{p_i}\rfloor)  $$

$$
p_i^2 >  j, g(i, j) = g(i-1, j) - p^k_i
$$

2. $$\sum_{\sqrt{n}  \le i \le n \&\& i \ has't \ bigger \ than \sqrt{n}}  F(i)$$

设f(i, j) 表示[1, j]中仅由i个质数个组成的数的F(x)之和
$$
f(i, j) = f(i-1, j) + \sum_{c \ge 1}F(p^c_i)f(i-1, \lfloor \frac{j}{p_i^c} \rfloor)
$$

$$
when\ p^2_i > j , f(i, j) = f(i-1, j) + F(p_i)
$$


{\bfseries 素数个数}

题意:求1-n中素数个数,n<1e11;

解析:对于这个虽然不是积性函数,但是我们依然可以分为两块1-$\sqrt{n}$ 和$(\sqrt{n}+1 ) -  n$

前者可以O(n)预处理,后者就可以利用上面第一块的$g(\sqrt{n},n)$递推

\begin{lstlisting}
#include <bits/stdc++.h>
const int N = 4e6+9;
using ll = long long;
using namespace std;
int prime[N], cnt, res[N]; bool vis[N];
...
ll n, v[N], k, last[N], g[N];
int solve(){
    if(n < N) return printf("%d\n", res[n]), 0;
    int tot = 0, sn = sqrt(n+0.5); // sqrt(n)
    int pos = upper_bound(prime, prime+cnt, sn) - prime; // `第一个大于sqrt(n)的质数`
    for (ll i = n; i >= 1; i = n/(n/i+1) ) v[++tot] = n/i; // `分块`
    for (int i = 1; i <= tot; i++) g[i]=v[i], last[i] = 0; // `初始化`
    for (int i = 0; i < pos; i++) {
        for (int j = tot; j; j--) {
            k = v[j]/prime[i]; if(k < prime[i]) break; // `忽略j小于p*p`
            k = k < sn? k: tot - n/k + 1; // `找到在v中的下标`
            g[j] -= g[k] - (i - last[k]); // `减去g[k]中已经减过的`
            last[j] = i + 1; // `上一次处理的是第i+1个质数(从1开始)`
        }
    } printf("%lld\n", res[sn]*1LL + g[tot] - 1);
}

int main(){
    init(); scanf("%lld", &n); return solve(), 0;
}
\end{lstlisting}

洲阁筛实现起来较为复杂,现在已经被Min\_25筛替代。

\subsection{其他杂题}

题意:定义w(n) = n的质因子个数,$g(n)=2^{w(n)}$, 求$S(n) = \sum_{i=1}^{n}g(i)\  mod\  1e9+7$,
CCPC 杭州2016J


\begin{eqnarray*} ans&=&\sum_{i=1}^ng(i) = \sum_{i=1}^n\sum_{d|i}\mu^2(d)\\ &=&\sum_{i=1}^n\sum_{d|i}\sum_{k^2|d}\mu(k)\\ &=&\sum_{k=1}^n\mu(k)\sum_{k^2|d}\lfloor\frac{n}{d}\rfloor\\ &=&\sum_{k=1}^n\mu(k)\sum_{i=1}^{\lfloor\frac{n}{k^2}\rfloor}\lfloor\frac{n}{k^2i}\rfloor\\ &=&\sum_{k=1}^{\sqrt{n}}\mu(k)S(\lfloor\frac{n}{k^2}\rfloor) \end{eqnarray*} 


\begin{lstlisting}
#include <bits/stdc++.h>
const int N = 1e6+9;
const int mod = 1e9+7;
using namespace std;
using ll = long long;
int prime[N], cnt, mu[N];
bool vis[N];
...
ll n;int f[N];
inline int S(ll n) {
    if(n < N && f[n]) return f[n];
    ll ans = 0;
    for (ll i = 1, j; i <= n; i = j+1) {
        j = n/(n/i); ans = (ans + (n/i)*(j-i+1))%mod;
    } if(n < N) f[n] = ans;
    return ans;
}

int main() {
    int T, ks=1; scanf("%d", &T); init();
    while(T--) {
        scanf("%lld", &n);
        ll ans = 0;
        for(ll k = 1; k * k <= n; k++) {
            if(mu[k]) ans = (ans + mu[k]*S(n/k/k)) % mod;
        } printf("Case #%d: %lld\n",ks++, (ans+mod)%mod);
    }
}
\end{lstlisting}


2018 四川省赛

题意:求$\sum_{i=1}^n\sum_{j=1}^i n \%(ij), n \le 1e11$

原式=$\sum_{i=1}^n\sum_{j=1}^i n - \lfloor \frac{n}{ij} \rfloor = n*n*(n+1)/2  -\sum_{i=1}^n\sum_{j=1}^i \lfloor \frac{n}{ij} \rfloor$

= $n*n*(n+1)/2  -1/2\sum_{i=1}^n\sum_{j=1}^n \lfloor \frac{n}{ij} \rfloor - 1/2\sum_{i=1}^n \lfloor \frac{n}{i^2} \rfloor $  

欧拉筛预处理前$n^{2/3}$项后其他的暴力算,总复杂度是$O(n^{2/3})$

由于最后答案较大需要使用大数或者\_\_int128

预处理的话,可以考虑$f[n] = \sum_{i=1}^i \lfloor \frac{n}{i}\rfloor i$, 那么$f[n]-f[n-1] = \sum_{i=1}^n i*( \lfloor \frac{n}{i}\rfloor -  \lfloor \frac{n-1}{i}\rfloor) = \sum_{i|n}i$, 这个在欧拉筛中可以预处理,然后求一遍前缀和

\begin{lstlisting}
#include <bits/stdc++.h>
#define ll long long
typedef __int128 dll;
const int N = 3e7;
using namespace std;
ll n;
int prime[N], cnt, t[N];
dll f[N];
bool vis[N]; 
void init(){
    f[1] = t[1] = 1;
    for (int i = 2; i < N; i++) {
        if(!vis[i]) {prime[cnt++]=i; f[i]=i+1; t[i]=i;}
        for (int j = 0; j < cnt; j++) {
            if(i*prime[j]>=N) break; vis[i*prime[j]] = 1;
            if(i%prime[j] == 0) {
                f[i*prime[j]] = f[i/t[i]] + f[i]*prime[j];
                t[i*prime[j]] = t[i] * prime[j]; break;
            } f[i*prime[j]] = f[i]*(prime[j]+1); t[i*prime[j]] = prime[j];
        }
    } for (int i = 2; i < N; i++) f[i] += f[i-1];
}
 
inline dll F(ll n) { // `分块求$\sum n/i$ * i`
    if(n < N) return f[n];
    dll ans = 0;
    for (ll i = 1,j; i <= n; i=j+1) {
        j = n/(n/i);
        ans += ((dll)(j-i+1))*(i+j)/2*(n/i);
    } return ans;
}
 
void print(dll x){ // `int128位数输出`
    if(x > 9) print(x/10);
    putchar(x%10 + '0');
}
 
int main() {
    int t; scanf("%d", &t); init();
    while(t--) {
        scanf("%lld", &n);
        dll ans = n; ans *= n; ans += ans * n; // ans = n*n*(n+1);
        for (int i = 1; 1LL * i * i <= n; i++) ans -= n/i/i*i*i;
        for (ll i = 1,j; i <= n; i=j+1) {
            j=n/(n/i);
            ans -= ((dll)(j-i+1))*(i+j)/2*F(n/i);
        } print(ans/2); puts("");
    } return 0;
}
\end{lstlisting}

\subsection{rng\_58-clj equation}

$$
\sum_{i=1}^a\sum_{j=1}^b\sum_{k=1}^cd(ijk) = \sum_{gcd(i,j)=gcd(j,k)=gcd(i,k)=1} \lfloor \frac{a}{i} \rfloor \lfloor \frac{b}{j} \rfloor \lfloor \frac{c}{k} \rfloor
$$


事实上这个等式可以扩展至任意维。

题意:求$\sum_{i=1}^a \sum_{j=1}^b$d(ij), T, a, b < 5e4;

原式化简为$\sum_{gcd(i,j)=1}\lfloor \frac{a}{i} \rfloor \lfloor \frac{b}{j} \rfloor = \sum_{g=1}^{min(a,b)} \mu(g) \sum_{i=1}^{a/g} \sum_{j=1}^{b/g}\lfloor \frac{a}{i} \rfloor \lfloor \frac{b}{j} \rfloor$

分块查询即可,代码略

题意:$\sum\limits_{i=1}^a\sum\limits_{j=1}^b\sum\limits_{k=1}^c d(ijk)$, a,b,c < 2000

化简:$\sum\limits_{i=1}^n\sum\limits_{j=1}^m\sum\limits_{k=1}^t {\lfloor {n\over i}\rfloor}{\lfloor {m\over j}\rfloor}{\lfloor {t\over k}\rfloor}  [(i,j)=1][(j,k)=1][(i,k)=1]$

=>$\sum\limits_{k=1}^t {\lfloor {t\over k}\rfloor} \sum\limits_{d=1}^n \mu(d) \sum\limits_{i=1}^ {\lfloor {n\over i}\rfloor}  {\lfloor {n\over i*d}\rfloor}[(di,k)=1] \sum\limits_{j=1}^ {\lfloor {m\over j}\rfloor} {\lfloor {m\over i*d}\rfloor} [(dj,k)=1]$

暴力算一下就行了(gcd可以记忆化一下)


\begin{lstlisting}
#include <bits/stdc++.h>
#define gcd __gcd
const int N = 2e3+5;
const int mod = (1<<30) - 1;
using namespace std;

int mu[N], prime[N], a, b, c, gd[N][N], cnt;
bool vis[N];
...
inline int cal(int d, int x){
    int ans = d;
    for(int i = 2; i <= d; i++) if(gd[i][x] == 1) ans += d/i;
    return ans;
}

int main(){
    scanf("%d%d%d",&a,&b,&c); init();
    int ans = 0; if(b > c) swap(b, c);
    for(int i = 1; i <= a; i++) { for(int j = 1; j <= b; j++)
        if(gcd(i, j) == 1 && mu[j]) ans += (a/i)*mu[j]*cal(b/j, i)*cal(c/j, i);
    } return printf("%u\n",ans&mod), 0;
}
\end{lstlisting}

{\bfseries BZOJ4176}

题意:$\sum_{i=1}^n \sum_{j=1}^nd(ij), n \le 1e9$

化简得$= \sum_{g=1}^{n} \mu(g) \sum_{i=1}^{n/g} \sum_{j=1}^{n/g}\lfloor \frac{n}{i} \rfloor \lfloor \frac{n}{j} \rfloor$

分块查询的难调在于u(g)的前缀和,这个使用杜教筛就可以解决。


\begin{lstlisting}
#include <bits/stdc++.h>
using namespace std;
using ll = long long;
const int N = 1e7+9;
const int mod = 1e9+7;
int prime[N], cnt;ll mu[N];
bool vis[N];
unordered_map<ll, ll> mp;
ll n;
...
ll dfs(ll n){ // `莫比乌斯前缀和`
    if(n < N) return mu[n];
    if(mp.count(n)) return mp[n];
    ll ans = 1; // e
    for (ll i = 2,j; i <= n; i=j+1){
        j = n/(n/i); ans -= dfs(n/i)*(j-i+1);
    } return mp[n]=ans;
}

ll F(ll n) { // `分块求和`
    ll ans = 0;
    for (ll i = 1,j; i <= n; i=j+1) {
        j = n/(n/i);
        ans = (ans + (j-i+1)%mod*(n/i)) % mod;
    } return ans * ans % mod;
}

int main() {
    init(); scanf("%lld", &n); ll ans = 0;
    for (ll i=1,j; i <= n; i=j+1) {
        j = n/(n/i);
        ans = (ans + (dfs(j) - dfs(i-1) + mod)%mod*F(n/i)%mod) % mod;
    } printf("%lld\n", ans);
}
\end{lstlisting}

\subsection{Min\_25筛}

Min\_25筛是一个替代洲阁晒的新产物,也是用于求解积性函数求和的问题,时间空间复杂度都比洲阁晒要优秀。

同样考虑筛质数,对1-n内所有质数求和。

设函数$S(x, j)=\sum_{i=2}^xi*[i$为质数或i的最小质因子大于$p_j]$ , 质数从p1开始

S(x, 0)=x*(x+1)/2 - 1; 我们要求的就是S(n, p0)

考虑转移:

$ p^2_j > x, S(x, j) = S(x, j-1) $

否则$ S(x,j)=S(x,j-1)-f(p_j)*(S(\lfloor \frac{x}{p_j} \rfloor,j-1)-\sum_{i=1}^{j-1}f(p_i)) $

求具体积性函数和时

$G(x,j)=G(x,j+1)+\sum_{k=1}(G(\lfloor\frac{x}{p_j^k}\rfloor,j+1)-\sum_{i=1}^xf(p_i) )*f(p_j^k)+\sum_{k=2}f(p_j^k)$

能力有限,以后再学。



section{原根}

每一个质数可以去求解其原根g,将原本的乘法运算转化为指数上的加法运算,可以利用FFT优化。


\begin{lstlisting}
int tp[50];
int find_root(int x) {
    int f, phi = x-1;
    for(int i = 0; phi && i < cnt; i++) {
        if(phi % prime[i] == 0) {
            tp[++tp[0]] = prime[i];
            while(phi%prime[i]==0) phi/=prime[i];
        }
    } for(int g = 2; g <= x-1; g++) {
        f = 1;
        for(int i=1;i<=tp[0];i++) {
            if(qp(g, (x-1)/tp[i], x)==1) {
                f=0; break;
            }
        } if(f) return g;
    } return 0;
}
\end{lstlisting}

{\bfseries 牛客camp day 2}

一个数列a ,2e5个数,输出n个数,表示存在多少数对(i,j) st.   $a_i * a_j \% p = a_k$

\begin{lstlisting}
#include <bits/stdc++.h>
using ll = long long;
using namespace std;
const int mod = 1e9+7;
const int N = 5e5+9;
int prime[N], p[N], cnt;
ll a[N], x[N], ans[N], e[N];
...
namespace FFT {
    #define rep(i,a,b) for(int i=(a);i<=(b);i++)
    const double pi=acos(-1);
    const int maxn=1<<19;
    struct cp {
        double a,b;
        cp(){}
        cp(double _x,double _y){a=_x,b=_y;}
        cp operator +(const cp &o)const{return (cp){a+o.a,b+o.b};}
        cp operator -(const cp &o)const{return (cp){a-o.a,b-o.b};}
        cp operator *(const cp &o)const{return (cp){a*o.a-b*o.b,b*o.a+a*o.b};}
        cp operator *(const double &o)const{return (cp){a*o,b*o};}
        cp operator !()const{return (cp){a,-b};}
    } x[maxn],y[maxn],z[maxn],w[maxn];
    void fft(cp x[],int k,int v) {
        int i,j,l;
        for(i=0,j=0;i<k;i++) {
            if(i>j)swap(x[i],x[j]);
            for(l=k>>1;(j^=l)<l;l>>=1);
        } w[0]=(cp){1,0};
        for(i=2;i<=k;i<<=1) {
            cp g=(cp){cos(2*pi/i),(v?-1:1)*sin(2*pi/i)};
            for(j=(i>>1);j>=0;j-=2)w[j]=w[j>>1];
            for(j=1;j<i>>1;j+=2)w[j]=w[j-1]*g;
            for(j=0;j<k;j+=i) {
                cp *a=x+j,*b=a+(i>>1);
                for(l=0;l<i>>1;l++) {
                    cp o=b[l]*w[l];
                    b[l]=a[l]-o;
                    a[l]=a[l]+o;
                }
            }
        } if(v)for(i=0;i<k;i++)x[i]=(cp){x[i].a/k,x[i].b/k};
    }
    void mul(ll *a,ll *b,ll *c,int l1,int l2) {
        if(l1<128&&l2<128) {
            rep(i,0,l1+l2)a[i]=0;
            rep(i,0,l1)rep(j,0,l2)a[i+j]+=b[i]*c[j];
            return;
        } int K;
        for(K=1;K<=l1+l2;K<<=1);
        rep(i,0,l1)x[i]=cp(b[i],0);
        rep(i,0,l2)y[i]=cp(c[i],0);
        rep(i,l1+1,K)x[i]=cp(0,0);
        rep(i,l2+1,K)y[i]=cp(0,0);
        fft(x,K,0);fft(y,K,0);
        rep(i,0,K)z[i]=x[i]*y[i];
        fft(z,K,1);
        rep(i,0,l1+l2)a[i]=(ll)(z[i].a+0.5);
    }
};

int main() {
    init(); int n, p, now, i, g;
    while(~scanf("%d%d",&n,&p)) {
        g = find_root(p); now = 1;
        for (int i = 1; i < p; i++) {
            now = now * g % p;
            e[now]=i%(p-1);
        } memset(a, 0, sizeof a);
        ll res = 0;
        for (int i = 1; i <= n; i++) {
            scanf("%d", &x[i]);
            if(x[i] % p) a[e[x[i]%p]]++;
            else res ++;
        } memset(ans, 0, sizeof ans);
        FFT::mul(ans, a, a, p-1, p-1);
        for (int i = p-1; i < 2*p; i++) {
            ans[i-(p-1)] += ans[i];
        } for (int i = 1; i <= n; i++) {
            if(x[i] >= p) puts("0");
            else if(x[i] == 0) printf("%lld\n", res*res + res*2*(n-res));
            else printf("%lld\n", ans[e[x[i]]]);
        }
    } return 0;
}
\end{lstlisting}

\subsection{二次剩余}

$x^2\equiv n\pmod p$ 我们只讨论p为奇素数的情况。 

$$
\left(\frac{a}{p}\right)=
\begin{cases}
1,&a\text{在模$p$意义下是二次剩余}\\
-1,&a\text{在模$p$意义下是非二次剩余}\\
0,&a\equiv0\pmod p
\end{cases}
$$

定理2:$\left(\frac{a}{p}\right)\equiv a^{\frac{p-1}{2}}\pmod p$

找到t使得$\left( \frac{t^2-n}{n} \right)=-1$,设 $\omega=\sqrt{t^2-n}$

解为$a\equiv(t+\omega)^{\frac{P+1}{2}} \pmod{P}$

未检验的模板:
\begin{lstlisting}
#define fo(i,a,b) for(int i=a;i<=b;++i)
#define fod(i,a,b) for(int i=a;i>=b;--i)
#define min(q,w) ((q)>(w)?(w):(q))
#define max(q,w) ((q)<(w)?(w):(q))
using namespace std;
typedef int LL;
const int N=1500;
int mo;
int read(int &n) {
    char ch=' ';int q=0,w=1;
    for(;(ch!='-')&&((ch<'0')||(ch>'9'));ch=getchar());
    if(ch=='-')w=-1,ch=getchar();
    for(;ch>='0' && ch<='9';ch=getchar())q=q*10+ch-48;n=q*w;return n;
}
int m,n,ans;
LL W;
struct qqww {
    LL x,y;
    qqww(LL _x=0,LL _y=0){x=_x,y=_y;}
    friend qqww operator *(qqww q,qqww w){return qqww((q.x*w.x+q.y*w.y%mo*W)%mo,(q.x*w.y+q.y*w.x)%mo);}
};
LL ksm(LL q,int w,int Mo) {
    LL ans=1;q=q%Mo;(q<0?q=q+Mo:0);
    for(;w;w>>=1,q=q*q%Mo)if(w&1)ans=ans*q%Mo;
    return ans;
}
qqww ksm(qqww q,int w,int Mo) {
    ::mo=Mo;
    qqww ans(1,0);
    for(;w;w>>=1,q=q*q)if(w&1)ans=ans*q;
    return ans;
}
int RD(int mo){return rand()%mo;}
LL Cipolla(int n,int mo) {
    if(w==2)return 1;
    LL q=ksm(n,(mo-1)>>1,mo);
    if(q==0||q==mo-1)return -1;
    for(q=RD(mo);ksm(q*q-n+mo,((mo-1)>>1),mo)!=mo-1;q=RD(mo));
    qqww t(q,1);W=(q*q-n+mo)%mo;
    t=ksm(t,((mo+1)>>1),mo);
    return (t.x+mo)%mo;
}
int main() {
    int q,w,_;
    srand(19890604);
    for(read(_);_;_--) {
        read(q),read(w);
        ans=Cipolla(q,w);
        if(ans==-1)printf("No root\n");
        else if(ans==w-ans)printf("%d\n",ans);
        else if(ans<w-ans)printf("%d %d\n",ans,w-ans);
        else printf("%d %d\n",w-ans,ans);
    } return 0;
}
\end{lstlisting}

\subsection{pell方程与BSGS}
pell 方程:
$x^2 - dy^2 = 1$ ,d为正整数

c++:

\begin{lstlisting}
bool pell(int D, int& x, int& y) {
    int sqrtD = sqrt(D + 0.4);
    if( sqrtD * sqrtD == D ) return false;
    int c = sqrtD, q = D - c * c, a = (c + sqrtD) / q;
    int step = 0;
    int X[] = { 1, sqrtD };
    int Y[] = { 0, 1 };
    while( true ) {
        X[step] = a * X[step^1] + X[step];
        Y[step] = a * Y[step^1] + Y[step];
        c = a * q - c;
        q = (D - c * c) / q;
        a = (c + sqrtD) / q;
        step ^= 1;
        if( c == sqrtD && q == 1 && step ) {
            x = X[0], y = Y[0];
            return true;
        }
    }
}
\end{lstlisting}

java:

\begin{lstlisting}
static class Pell {
    int D;
    BigInteger x, y ;
    boolean status = true;
    Pell() {};
    Pell(int d) {D = d;}
    void slove() {
        int sqrtD = (int)Math.sqrt((double)D);
        if(sqrtD * sqrtD == D){ status = false; return ; }
        BigInteger N = BigInteger.valueOf(D);
        BigInteger SqrtD = BigInteger.valueOf(sqrtD);
        BigInteger c = SqrtD;
        BigInteger q = N.subtract(c.multiply(c));
        BigInteger a = c.add(SqrtD).divide(q);

        int step = 0;
        BigInteger[] X = {BigInteger.ONE, SqrtD};
        BigInteger[] Y = {BigInteger.ZERO, BigInteger.ONE};
        while(true) {
            X[step] = a.multiply(X[step ^ 1]).add(X[step]);
            Y[step] = a.multiply(Y[step ^ 1]).add(Y[step]);

            c = a.multiply(q).subtract(c);
            q = (N.subtract(c.multiply(c))).divide(q);
            a = (c.add(SqrtD)).divide(q);
            step ^= 1;
            if (c.equals(SqrtD) && q.equals(BigInteger.ONE) && step == 1) {
                x = X[0]; y = Y[0]; return;
            }
        }
    }
}       
\end{lstlisting}

java : 连分数法(答案很大)

\begin{lstlisting}
/*
 * `逐项求解sqrt(d)的连分数`
 */
BigInteger x = BigInteger.ONE;
BigInteger y = BigInteger.ONE;
BigInteger a,ak,N,P1,P2,p1,p2,Q1,Q2,q1,q2;
q1 = p2 = P1 = BigInteger.ZERO;
p1 = q2 = Q1 = BigInteger.ONE;
N =  BigInteger.valueOf(n);
a = BigInteger.valueOf(k);
ak = a;
while(!x.multiply(x).subtract(N.multiply(y).multiply(y)).equals(BigInteger.ONE)){
    x = ak.multiply(p1).add(p2);
    y = ak.multiply(q1).add(q2);
    P2 = ak.multiply(Q1).subtract(P1);
    Q2 = N.subtract(P2.multiply(P2)).divide(Q1);
    ak = P2.add(a).divide(Q2);

    P1 = P2; Q1 = Q2;

    p2 = p1; p1 = x;
    q2 = q1; q1 = y;
}
\end{lstlisting}

\subsubsection{BSGS}
bsgs大步小步算法解决: $a^x \equiv b (mod \ p)$ ,算x

\begin{lstlisting}
const int mod = 1e5 + 7;
ll has[mod + 100], id[mod + 100];
ll find(ll x){
    ll t = x % mod;
    while(has[t] != x && has[t] != -1){ t = (t+1)%mod; }
    return t;
}
void insert(ll x, ll i){
    ll pos = find(x);
    if(has[pos] == -1){ has[pos] = x; id[pos] = i; }
}
ll get(ll x){
    ll pos = find(x);
    return has[pos] == x? id[pos] : -1;
}
void ex_gcd(ll a, ll b, ll &d, ll &x, ll &y){
    if(b == 0){ x = 1; y = 0; d = a; return ; }
    ex_gcd(b, a%b, d, y, x); y -= a/b*x;
}
ll inv(ll a, ll p){
    ll x, y ,d; ex_gcd(a, p, d, x, y);
    return d==1?(x%p+p)%p : -1;
}

ll BSGS(ll a, ll b, ll p){
    memset(has, -1, sizeof has);
    memset(id, -1, sizeof id);
    ll m = (ll) ceil(sqrt(p+0.5)); ll tmp = 1;
    for(ll i = 0; i < m; i++){
        insert(tmp , i);
        tmp = tmp * a % p;
    } ll base = inv(tmp, p);
    if(base == -1) return -1;
    ll res = b, z;
    for(ll i = 0; i < m; i++){
        if((z = get(res)) != -1) return i * m + z;
        res = res * base % p;
    } return -1;
}
\end{lstlisting}

扩展ex\_bsgs, 模数任意

\begin{lstlisting}
ll slove(ll a, ll b, ll p){
    ll tmp = 1;
    for(int i = 0; i < 50; i++) {
        if(tmp == b) return i;
        tmp = tmp * a % p;
    } ll cnt = 0, d = 1 % p;
    while((tmp = gcd(a, p)) != 1){
        if(b % tmp)return -1;
        b /= tmp; p /= tmp;
        d = a / tmp * d % p;
        cnt ++ ;
    } b = b * inv(d, p) % p;
    ll ans = BSGS(a, b, p);
    if(ans == -1)return -1;
    else return ans + cnt;
}

\end{lstlisting}

\subsection{扩展Lucas}

解决$C(n,m) \% p $, p 不为质数的情况

\begin{lstlisting}
ll C(ll n, ll m, ll p) {
    if(m>n) return 0;
    ll res = 1, i, a, b;
    for(i = 1; i <= m; i++) {
        a = (n+1-i) % p;
        b = inv(i%p, p);
        res = res*a%p*b%p;
    } return res;
}

ll Lucas(ll n, ll m, ll p) {
    if(m == 0) return 1;
    return Lucas(n/p, m/p, p)*C(n%p, m%p, p) % p;
}

ll cal(ll n, ll a, ll b, ll p) {
    if(!n) return 1;
    ll i, y = n/p, tmp = 1;
    for(i = 1; i <= p; i++) if(i%a) tmp = tmp*i%p;
    ll ans = pow(tmp, y, p);
    for(i = y*p+1; i <= n; i++) if(i%a) ans = ans*i%p;
    return ans * cal(n/a, a, b, p)%p;
}

ll multiLucas(ll n, ll m, ll a, ll b, ll p) {
    ll i, t1, t2, t3, s = 0, tmp;
    for(i = n; i; i/=a) s += i/a;
    for(i = m; i; i/=a) s -= i/a;
    for(i = n-m; i; i/=a) s -= i/a;
    tmp = pow(a, s, p);
    t1 = cal(n, a, b, p);
    t2 = cal(m, a, b, p);
    t3 = cal(n-m, a, b, p);
    return tmp*t1%p*inv(t2, p)%p*inv(t3, p)%p;
}

ll exLucas(ll n, ll m, ll p) {
    ll i, d, c, t, x, y, q[100], a[100], e = 0;
    for(i = 2; i*i <= p; i++) {
        if(p % i == 0) {
            q[++e] = 1;
            t = 0;
            while(p%i==0) {
                p /= i;
                q[e] *= i;
                t++;
            }
            if(q[e] == i) a[e] = Lucas(n, m, q[e]);
            else a[e] = multiLucas(n, m, i, t, q[e]);
        }
    }
    if(p > 1) {
        q[++e] = p;
        a[e] = Lucas(n, m, p);
    }
    for(i = 2; i <= e; i++) {
        d = exgcd(q[1], q[i], x, y);
        c = a[i]-a[1];
        if(c%d) exit(-1);
        t = q[i]/d;
        x = (c/d*x%t+t)%t;
        a[1] = q[1]*x+a[1];
        q[1] = q[1]*q[i]/d;
    }
    return a[1];
}
\end{lstlisting}



\subsection{关于模数非素数的问题}


关于模数为合数的情况,我们通常采用几种情况:

1. 使用java种的BigIntegeter

2. 使用自带的c++大数模板或者\_\_int128

3. 如果除数为n, 那么就讲模数改为n*mod, 得到的最后答案处以n,这个情况就要求你最后的数是一定能够被n 整除的。与此同时,还要注意一个事项模数大了之后,乘法运算会溢出。

采用快速乘法, 或者使用:

\begin{lstlisting}
ll qmul(ll x, ll y) {
    return (x*y - (ll)((long double)x/mod*y)*mod+mod)%mod
}
\end{lstlisting}
