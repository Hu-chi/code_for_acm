\subsection{Hash}


有时候我们要统计很大的数的个数时,但是这个时候我们又开不下这么大的数组,我们常常可以借助map,或者$hash_table$之类的工具。

Hash 统计数字个数

有时候map占用的数据也非常大, 我们可以利用链式前向星来代替链表实现一个简易的hash来统计每个数的个数,这个所占用的空间比map要小的多。

\subsubsection{统计个数}

\begin{lstlisting}
const int mod=1<<15;
struct Edge{int v,nxt,w;}e[mod];
int head[mod], cnt;
void init(){ memset(head, -1, sizeof head); cnt = 0; }

int gethash(int x){ // `将数字hash`
    return (x+ mod) & (mod-1);
}

void insert(int x) { // `插入`
    int id=gethash(x); // `得到hash值`
    for(int i=head[id]; i != -1;i=e[i].nxt) { // `便利冲突的元素`
        if(e[i].v==x) { e[i].w++; return; }
    } e[cnt] = {x, head[id], 1}; head[id]=cnt++; // `新元素额外添加`
}

void search(int x) { // `查找个数`
    int id=gethash(x);
    for(int i=head[id] ; i!=-1;i=e[i].nxt) {
        if(e[i].v==x) return e[i].w;
    } return 0;
}
\end{lstlisting}

\subsection{统计字符串}

如果我们就考虑小写字母的话,那么有26种小写字母,那么对于1e6长度的字符串,就有$10^{1414973}$多种情况,而对于我们所要处理的字符串实在是太少,所以我们对于字符串hash很少来处理冲突,如果发现确实有冲突,那么我们可以采用使用两种不同的hash来降低同时冲突的概率。

一般我们对于字符串是如此hash的:

比如有一个字符串abc

我们构造一个函数:$f(x) = a[0]*x^{n-1} + a[1]*x^{n-2} + ... + a[n-1]$

x就是我们自己选取的数,一般常用一个质数,比如131,233之类的也用字母种类数比如26之类的。

那么对于上面这个字符串$f(26) = a*x^{2} + b*x^{2} + c$

a,b,c的值可以是1,2,3或者是他们本身的ASCII值。

使用这种方式的好处就是一次O(n)的预处理后可以O(1)查询任意字串的值,除此之外还有滚动哈希之类的操作.

\subsection{子串种类}

题意: 给定一个字符串s, 查询长度为m的子串一共有多少种.

{\bfseries 滚动hash}

滚动hash其实就是我们先计算第一个长度为m的字符串的hash值,我们去计算第二个长度为m的子串时,我们只需要使用上一个hash值减去其第一个字母的影响然后乘x再加上后面缺少的那个字母的值.  举例如下

$f(x) = a[0]*x^{n-1} + a[1]*x^{n-2} + ... + a[n-1] $ , 我们先减去 $a[0] * x^{n-1} $ 然后整体乘x , 最后加上a[n].

\begin{lstlisting}
#include <cstdio>
#include <cstring>
const int N = 1.6e7+9;
using namespace std;
char s[N];
bool has[N], vis[256];
int n, nc, a[256], m, cnt;

int main(){
    scanf("%d%*d%s",&m,s); n = strlen(s);
    for (int i = 0; i < n; i++) {
        if(!vis[s[i]]) {
            vis[s[i]] = 1; a[s[i]]=nc++; // `对于字母出现的顺序来决定大小`
        }
    } int ans = 0, x = 0, y = 1;
    for (int i = 0; i < m; i ++) {
        x = x*nc + a[s[i]];         // `计算第一个长度为m的字符串的hash值`
        y *= nc;
    } y/= nc; has[x] = 1; ans ++; // y = nc^{m-1}
    for (int i = 1; i <= n-m; i++) {
        x = (x - a[s[i-1]]*y) * nc + a[s[i+m-1]]; // `滚动hash`
        if(!has[x]) has[x] = 1, ans++;
    } return printf("%d\n", ans), 0;
}
\end{lstlisting}

{\bfseries O(n)预处理后整体查询}

由于map的速度实在是太慢了,而且占用空间也大,我们使用之前自己写的.

这里我们o(n)预处理的h(i) 代表的是 以i为结尾的 字串的hash值. bit则存放的是$x^i$, 那么我们想要得到一个字串(l,r)的值的话我们对于0-r的字串的值减去前0-(l-1) 的字串的影响就行了.

\begin{lstlisting}
#include <cstdio>
#include <map>
#include <cstring>
#define ull unsigned long long
const int base = 1e9+9;
const int N = 2e6+9;
using namespace std;
int n, m;
ull bit[N], h[N];
char s[N];

const int mod=1<<15;
struct Edge{int v,nxt,w;}e[mod];
int head[mod], cnt;
void init(){memset(head, -1, sizeof head); cnt = 0;}
int gethash(int x){return (x+ mod) & (mod-1);}
void insert(int x) {
    int id=gethash(x); 
    for(int i=head[id]; i != -1;i=e[i].nxt) {
        if(e[i].v==x) { e[i].w++; return;}
    } e[cnt].v = x; e[cnt].nxt = head[id]; e[cnt].w = 1; head[id]=cnt++;
}

int main(){
    memset(head, -1, sizeof head);    // `初始化`
    scanf("%d%*d%s", &m, s+1); bit[0] = 1; n = strlen(s+1);
    for (int i = 1; i < N; i++) bit[i] = bit[i-1]*base;      // x^i
    for (int i = 1; i <= n; i++) h[i] = h[i-1]*base + s[i];  // `以i为结尾的 字串的hash值`
    for (int i = m; i <= n; i++) {
        ull tmp = (h[i] - h[i-m]*bit[m]);                    // `i-m+1, i 字串的hash值`
        insert(tmp);
    } printf("%d\n", cnt);
    return 0;
}
\end{lstlisting}

{\bfseries 二维字符矩阵}

题意: 给出一个n * m的矩阵。让你从中发现一个最大的正方形。使得这样子的正方形在矩阵中出现了至少两次。输出最大正方形的边长。  

$$
f(x, y ) = a[0][0] * x^{m-1} * y^{n-1} + a[0][1] * x^{m-2}*y^{n-1} + ... + a[0][m-1]*y^{n-1} \\
\  \ \ \ \ \ \ \ \ \ \ \ + a[1][0] * x^{m-1} * y^{n-2} + a[1][1] * x^{m-2}*y^{n-2} + ... + a[1][m-1]*y^{n-2} \\
+ ...   \ \ \ \ \ \ \ \ \ \ \ \ \ \ \ \ \ \ \ \ \ \ \ \ \ \ \ \ \ \ \ \ \ \ \ \ \ \ \ \ \ \ \ \ \ \ \ \ \ \ \ \ \ \ \ \ \ \ \ \ \ \ \ \ \ \ \ \ \ \ \ \ \ \ \ \ \ \ \ \ \ \ \ \ \ \ \ \ \ \ \ \ \ \ \ \ \ \ \
\\   + a[n-1][0] * x^{m-1} + a[n-1][1] * x^{m-2} + ... + a[n-1][m-1]
$$

这样下来我们如果要计算一个左上角(l1, r1) ,右下角为(l2, r2)的字符矩阵的hash值就方便的多了.

如果x边长的正方形矩阵满足题意,那么1- (x-1)边长的正方形矩阵都满足,所以具有二分性质,我们可以使用二分来枚举边长

\begin{lstlisting}
#include <bits/stdc++.h>

using namespace std;
using ull = unsigned long long;
const int N = 555;
const int mod = 1e5+7;
struct Edge { ull v; int nxt; }e[N*N];
int head[mod];
ull base1 = 1313, base2 =13131;
ull has1[N][N], has2[N][N], bit1[N], bit2[N];
char s[N][N];
int n, m, cnt;

void init(){    // `初始化`
    bit1[0] = bit2[0] = 1; cnt = 0;
    for (int i = 1; i < N; i++) {
        bit1[i] = bit1[i-1]*base1;
        bit2[i] = bit2[i-1]*base2;
    } for (int i = 1; i <= n; i++) {
        for (int j = 1; j <= m; j++) {
            has1[i][j] = has1[i][j-1]*base1+s[i][j]; // `计算的是i行的以j为末尾的字符串的hash值`
            has2[i][j] = has2[i-1][j]*base2+has1[i][j]; // `计算$1-i$, $1-j$ 的矩阵的hash值`
        }
    }
}

void inser(int u, ull v){ e[cnt].v = v; e[cnt].nxt = head[u]; head[u] = cnt++;}

ull cal(int i, int j, int x){  // `计算 (i,j) (i+x-1, j+x-1) 矩阵的hash值`
    ull ret = has2[i+x-1][j+x-1];    
    ret -= has2[i+x-1][j-1] * bit1[x] + has2[i-1][j+x-1]*bit2[x];
    ret += has2[i-1][j-1] * bit1[x] * bit2[x];
    return ret;
}

bool sech(ull x){
    int t = x % mod;
    for (int i = head[t]; ~i; i = e[i].nxt) {
        if(e[i].v == x) return 1;
    } return 0;
}

bool check(int x){  // `判断是否存在满足题意的 x长度的边长的矩阵`
    cnt = 0; memset(head, -1, sizeof head); // `清空hash表种的元素`
    for (int i = 1; i+x-1 <= n; i++) {
        for (int j = 1; j+x-1 <= m; j++) {
            ull t = cal(i, j, x); // `每次计算一个矩阵`
            if(sech(t)) return 1; // `查询是否有同样的矩阵`
            inser(t%mod, t);      // `没有就将矩阵的hash值插入`
        }
    } return 0;
}

int main() {
    scanf("%d%d", &n, &m);  
    for (int i = 1; i <= n; i++) {
        scanf("%s", s[i]+1);
    } init(); int l = 1, r = min(m, n); 
    while(l <= r){      // `二分边长`
        int mid = (l+r)>>1;
        if(check(mid)) l = mid + 1;
        else r = mid - 1;
    } return printf("%d\n", l-1), 0;
}
\end{lstlisting}

{\bfseries 最小表示法}

求字符串的循环表示法:

我们有一个字符串s,我们可以不停的把头上的字母放在尾巴上这样就构成一个循环串s[k],s[k+1],s[k+2],..,s[n],s[1],..,s[k-1]; 我们可以得到n个循环串,我们要去寻找字典序最小的串的首字母的原始下标。

我们可以使用双指针法,可以做到O(n), 也可以利用hash和二分算法不过这个复杂度是O(nlogn)的;

\begin{lstlisting}
int getMi(){
    int i = 0, j = 1, l; // `i,j 就是需要比较的两个循环串的首字母下标,l是长度`
    while(i < n && j < n){
        for(l = 0; l < n; l++){
            if(s[(i+l)%n] != s[(j+l)%n]) break;  // `找到第一个不相等的。`
        } if(l >= n) break; // `没找到说明两个循环串相等`
        if(s[(i+l)%n] < s[(j+l)%n]) j += l+1; // `如果i串更小,那么j移动l+1位,这里改成>就是字典序最大`
        else i += l+1;                        // `否则i串移动l+1位`
        if(i==j) j++;    // `如果i==j了让j往后移动一格`
    } return i<j?i:j;   // `返回下标小的(此时要么有一个>n或者i,j串相等)`
}
\end{lstlisting}

{\bfseries 二分+ Hash}

hash的做法是n个字符串我们从第一个开始比较,始终维护最小的循环串,比较字符串是log(n)比较的,方法是二分到两个字符串第一个不同的位置。所以总体复杂度是O(nlogn)

\begin{lstlisting}
#include <bits/stdc++.h>
using namespace std;
using ull = unsigned long long;
const int N = 1e5+9;
const int base = 131;
int n;
char s[N];
ull bit[N], has[N];

bool check(int x, int y, int l){ // `判断x,y开始l长度的字符串是否相等`
    ull u = has[x+l-1] - has[x-1]*bit[l];
    ull v = has[y+l-1] - has[y-1]*bit[l];
    return u == v;
}

bool ok(int x, int y){ // `比较那个下标开始的串更小`
    int l = 0, r = n;
    while(l <= r) {
        int mid = (l+r) >> 1;
        if(check(x, y, mid)) l = mid + 1;
        else r = mid - 1;
    } if(l-1==n) return 0;
    return (s[x+l-1]) < (s[y+l-1]);
}

int main() {
    scanf("%s", s+1); bit[0] = 1; n = strlen(s+1);
    for (int i = 1; i <= n; i++) s[i+n] = s[i];
    for (int i = 1; i < N; i++) bit[i] = bit[i-1]*base;
    for (int i = 1; i <= n*2; i++) has[i] = has[i-1]*base + s[i];
    int x = 1;
    for (int i = 2; i <= n; i++) {
        if(ok(i, x)) x = i;
    } return printf("%d\n", x), 0;
}
\end{lstlisting}


\subsection{Trie树}

01 字典树

\begin{lstlisting}
namespace trie{
    const int NODE = 2e6 * 2;
    int ch[NODE][2], size[NODE], pool = 2;
    // int vis[NODE], vid = 1;

    inline void init(){
        memset(ch, 0, sizeof ch);
        memset(size, 0, sizeof size);
    }

    inline void ins(bs a, bool flag) {
        int u = 1;
        for (int i = 31; i >= 0; i--) {
            if (!ch[u][a[i]]) ch[u][a[i]] = pool++;
            u = ch[u][a[i]];
            if (flag) size[u] ++;
            else size[u] --;
        }
    }

    inline ll find(bs a) {
        int u = 1; ll ans = 0;
        for (int i = 31; i >= 0; i--) {
            if (ch[u][1-a[i]] && size[ch[u][1-a[i]]]) {
                ans |= (1LL << i );
                u = ch[u][1-a[i]];
            } else {
                u = ch[u][a[i]];
            }
        } return ans;
    }
};
\end{lstlisting}

可持久化字典树,区间查询


\begin{lstlisting}
#define bs bitset<32>
const int N = 6e5+9;
namespace trie{
    int ch[N<<5][2], root[N], size[N<<5], pool = 1;
    bs tmp;

    inline void ins(int &nw, int od, int idx=24) {
        nw = pool++;
        for (int i = 0; i < 2; i++) ch[nw][i] = ch[od][i];
        size[nw] = size[od] + 1;
        if(!idx) return ;
        int x = tmp[idx-1];
        ins(ch[nw][x], ch[od][x], idx-1);
    }

    inline int query(int l, int r, int idx=24) {
        if(!idx) return 0;
        int x = tmp[idx-1];
        if(size[ch[r][x]] > size[ch[l][x]])
            return query(ch[l][x], ch[r][x], idx-1) + (1<<(idx-1));
        return query(ch[l][1-x], ch[r][1-x], idx-1);
    }
};
\end{lstlisting}


带延迟操作的01字典树合并-csl

\begin{lstlisting}
#include <bits/stdc++.h>
using namespace std;

const int maxn = 2e7;
struct Node {
    int sz, tag;
    Node* ch[2];
    Node() : sz(0), tag(0) { memset(ch, 0, sizeof(ch)); }
    void maintain() {
        sz = 0;
        if (ch[0] != nullptr) sz += ch[0]->sz;
        if (ch[1] != nullptr) sz += ch[1]->sz;
    }
    void pushdown() {
        if (tag == 0) return;
        if (tag & 1) {
            swap(ch[0], ch[1]);
            if (ch[0] != nullptr) ch[0]->tag++;
        }
        if (tag > 1) {
            if (ch[0] != nullptr) ch[0]->tag += (tag >> 1);
            if (ch[1] != nullptr) ch[1]->tag += (tag >> 1);
        } tag = 0;
    }
} mem[maxn];

Node* newNode() {
    static int tot = 0;
    assert(tot + 1 < maxn);
    return &mem[tot++];
};

struct Trie {
    void insert(Node*& o, int x, int dep = 30) {
        if (o == nullptr) o = newNode();
        if (dep == 0) {
            o->sz++;
            return;
        }
        o->pushdown();
        insert(o->ch[x & 1], x >> 1, dep - 1);
        o->maintain();
    }
    Node* merge(Node* o, Node* k, int dep = 30) {
        if (o == nullptr) return k;
        if (k == nullptr) return o;
        if (o == nullptr && k == nullptr) return nullptr;
        if (dep == 0) {
            o->sz += k->sz;
            return o;
        }
        o->pushdown(), k->pushdown();
        o->ch[0] = merge(o->ch[0], k->ch[0], dep - 1);
        o->ch[1] = merge(o->ch[1], k->ch[1], dep - 1);
        o->maintain();
        return o;
    }
    int query(Node* o, int x, int dep) {
        if (o == nullptr) return 0;
        if (dep == 0) return o->sz;
        o->pushdown();
        return query(o->ch[x & 1], x >> 1, dep - 1);
    }
};

struct DSU {
    vector<int> fa;
    DSU() {}
    DSU(int n) { fa.resize(n), iota(fa.begin(), fa.end(), 0); }
    int find(int x) { return fa[x] == x ? x : fa[x] = find(fa[x]); }
    bool same(int x, int y) { return find(x) == find(y); }
    void merge(int x, int y) { fa[find(y)] = find(x); }
};

int main() {
    int n, m;
    scanf("%d%d", &n, &m);
    DSU d(n);
    Trie t;
    vector<Node*> rt(n);
    for (int i = 0, x; i < n; i++) {
        scanf("%d", &x);
        t.insert(rt[i], x);
    }
    while (m--) {
        static int op, u, v, k, x;
        scanf("%d", &op);
        if (op == 1) {
            scanf("%d%d", &u, &v);
            --u, --v;
            if (!d.same(u, v)) {
                rt[d.find(u)] = t.merge(rt[d.find(u)], rt[d.find(v)]);
                d.merge(u, v);
            }
        } else if (op == 2) {
            scanf("%d", &u); --u;
            rt[d.find(u)]->tag++;
        } else {
            scanf("%d%d%d", &u, &k, &x); --u;
            printf("%d\n", t.query(rt[d.find(u)], x, k));
        }
    } return 0;
}
\end{lstlisting}

\subsection{KMP}

\begin{lstlisting}
void getNxt(){
    int i = 0, j = -1;
    nxt[i] = -1;
    while(i<m){
        if(j==-1||p[i]==p[j]){
            i++; j++; nxt[i] = j;
        }else j = nxt[j];
    }
}
\end{lstlisting}


\subsection{马拉车}
\begin{lstlisting}
char s[maxn];
int p[maxn];
int Manacher(){
    int len = strlen(s),id=0,r=0;
    for(int i = len; i >= 0; i--){
        s[i*2+2] = s[i];
        s[i*2+1] = '#';
    }
    s[0] = '#';
    for(int i = 2; i < 2*len + 1; i++){
        if(p[id] + id>i)p[i] = min(p[2*id - i], p[id]+id-i);
        else p[i] = 1;
        while(s[i-p[i]] == s[i+p[i]]) ++ p[i];
        if(id + p[id] < i + p[i])id = i;
        if(r < p[i]) r = p[i];
    }
    return r - 1;
}
\end{lstlisting}

\subsection{AC自动机}

题意:第一行输入测试数据的组数,然后输入一个整数n,接下来的n行每行输入一个单词,最后输入一个字符串,问在这个字符串中有多少个单词出现过。

\begin{lstlisting}
#include <bits/stdc++.h>
using namespace std;
const int maxn = 1e6;
const int maxk = 26;
struct Node{
    int nxt[26];
    int fail;
    int cnt;
    void init(){
        memset(nxt, -1, sizeof nxt);
        fail = -1;
        cnt = 0;
    }
} trie[maxn];
int root, num;
int que[maxn], head, tail;
char w[60], str[maxn];

void init(){ root=num=0; trie[root].init(); }

inline void inser(char *s){
    int idx, p = root;
    while(*s){
        idx = *s - 'a';
        if(trie[p].nxt[idx]==-1){
            trie[++num].init();
            trie[p].nxt[idx] = num;
        }
        p = trie[p].nxt[idx];
        s ++;
    }
    trie[p].cnt++;
}

void AC(){
    int cur,p,son;
    head = tail = 0;
    que[tail++] = root;
    while(head < tail){
        cur = que[head++];
        for(int i = 0; i < maxk; i++){
            if(trie[cur].nxt[i]!=-1){
                son = trie[cur].nxt[i];
                p = trie[cur].fail;
                if(cur == root) trie[son].fail = root;
                else trie[son].fail = trie[p].nxt[i];
                que[tail++] = son;
            } else {
                p = trie[cur].fail;
                if(cur == root){
                    trie[cur].nxt[i] = root;
                }
                else trie[cur].nxt[i] = trie[p].nxt[i];
            }
        }
    }
}

int Find(char *s){
    int p = root, cnt = 0, idx, tmp;
    while(*s){
        idx = *s - 'a';
        p = trie[p].nxt[idx];
        if(p==-1) p = root;
        tmp = p;
        while(tmp != root && trie[tmp].cnt != -1){
            cnt += trie[tmp].cnt;
            trie[tmp].cnt = -1;
            tmp = trie[tmp].fail;
        }
        s ++;
    }
    return cnt;
}
int n;
int main(){
    int t;scanf("%d",&t);
    while(t--){
        scanf("%d",&n);
        init();
        getchar();
        for(int i = 0; i < n; i++){
            gets(w);
            inser(w);
        }
        AC();
        gets(str);
        printf("%d\n",Find(str));
    }
}
\end{lstlisting}
