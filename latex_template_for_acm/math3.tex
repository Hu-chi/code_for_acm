\subsection{数值计算}

\subsubsection{Simpson积分}

\begin{lstlisting}
double f(double x) { return x * x + x;//`写要求辛普森积分的函数` }

double simpson(double L, double R) {//`三点辛普森积分法,要求f(x)是全局函数`
    double mid = (L + R) / 2.0;
    return (f(L) + 4.0 * f(mid) + f(R)) * (R - L) / 6.0;
}

double integral(double L, double R, double Eps){//`自适应辛普森积分递归过程`
    double mid = (L + R) / 2.0;
    double ST = simpson(L, R), SL = simpson(L, mid), SR = simpson(mid, R);
    if(fabs(SL + SR - ST) <= 15.0 * Eps)  return SL + SR + (SL + SR - ST) / 15.0;//`直接返回结果`
    return integral(L, mid, Eps/2.0) + integral(mid, R, Eps/2.0);//`对半划分区间`
}
\end{lstlisting}
红书Simpson 积分,红书还拥有Romberg积分

\begin{lstlisting}
template<class T>
double simpson(const T&f, double a, double int) {
    const double h = (b-a)/n;
    double ans = f(a) + f(b);
    for (int i = 1; i < n; i+=2) ans += 4*f(a+i*h);
    for (int i = 2; i < n; i+=2) ans += 2*f(a+i*h);
    return ans * h / 3;
}
\end{lstlisting}

\subsubsection{拉格朗日插值}

\begin{lstlisting}
namespace polysum {
	#define rep(i,a,n) for (int i=a;i<n;i++)
	#define per(i,a,n) for (int i=n-1;i>=a;i--)
	const int D=2010;
	ll a[D],f[D],g[D],p[D],p1[D],p2[D],b[D],h[D][2],C[D];
	ll powmod(ll a,ll b){ll res=1;a%=mod;assert(b>=0);for(;b;b>>=1){if(b&1)res=res*a%mod;a=a*a%mod;}return res;}
	ll calcn(int d,ll *a,ll n) { // a[0].. a[d]  a[n]
		if (n<=d) return a[n];
		p1[0]=p2[0]=1;
		rep(i,0,d+1) {
			ll t=(n-i+mod)%mod;
			p1[i+1]=p1[i]*t%mod;
		}
		rep(i,0,d+1) {
			ll t=(n-d+i+mod)%mod;
			p2[i+1]=p2[i]*t%mod;
		}
		ll ans=0;
		rep(i,0,d+1) {
			ll t=g[i]*g[d-i]%mod*p1[i]%mod*p2[d-i]%mod*a[i]%mod;
			if ((d-i)&1) ans=(ans-t+mod)%mod;
			else ans=(ans+t)%mod;
		}
		return ans;
	}
	void init(int M) {
		f[0]=f[1]=g[0]=g[1]=1;
		rep(i,2,M+5) f[i]=f[i-1]*i%mod;
		g[M+4]=powmod(f[M+4],mod-2);
		per(i,1,M+4) g[i]=g[i+1]*(i+1)%mod;
	}
	ll polysum(ll m,ll *a,ll n) { // a[0].. a[m] `$\sum_{i=0}^{n-1} a[i]$`
		ll b[D];
		for(int i=0;i<=m;i++) b[i]=a[i];
		b[m+1]=calcn(m,b,m+1);
		rep(i,1,m+2) b[i]=(b[i-1]+b[i])%mod;
		return calcn(m+1,b,n-1);
	}
	ll qpolysum(ll R,ll n,ll *a,ll m) { // a[0].. a[n] `$\sum_{i=0}^{m-1} a[i]*R^i$`
		if (R==1) return polysum(n,a,m);
		a[m+1]=calcn(m,a,m+1);
		ll r=powmod(R,mod-2),p3=0,p4=0,c,ans;
		h[0][0]=0;h[0][1]=1;
		rep(i,1,m+2) {
			h[i][0]=(h[i-1][0]+a[i-1])*r%mod;
			h[i][1]=h[i-1][1]*r%mod;
		}
		rep(i,0,m+2) {
			ll t=g[i]*g[m+1-i]%mod;
			if (i&1) p3=((p3-h[i][0]*t)%mod+mod)%mod,p4=((p4-h[i][1]*t)%mod+mod)%mod;
			else p3=(p3+h[i][0]*t)%mod,p4=(p4+h[i][1]*t)%mod;
		}
		c=powmod(p4,mod-2)*(mod-p3)%mod;
		rep(i,0,m+2) h[i][0]=(h[i][0]+h[i][1]*c)%mod;
		rep(i,0,m+2) C[i]=h[i][0];
		ans=(calcn(m,C,n)*powmod(R,n)-c)%mod;
		if (ans<0) ans+=mod;
		return ans;
	}
} // polysum::init();

\end{lstlisting}



\subsection{牛顿迭代开方}

使用牛顿迭代开方比二分快许多
\begin{lstlisting}
double NewtonMethod(double fToBeSqrted) {
    double x = 1.0;
    while(abs(x*x-fToBeSqrted) > 1e-5) x = (x+fToBeSqrted/x)/2;
    return x;
}
\end{lstlisting}

快速求解开方的倒数(真的很快):
\begin{lstlisting}
float Q_rsqrt( float number ) {
   long i;
   float x2, y;
   const float threehalfs = 1.5F;

   x2 = number * 0.5F;
   y   = number;
   i   = * ( long * ) &y;   // evil floating point bit level hacking
   i   = 0x5f3759df - ( i >> 1 ); // what the fuck?
   y   = * ( float * ) &i;
   y   = y * ( threehalfs - ( x2 * y * y ) ); // 1st iteration
   // y   = y * ( threehalfs - ( x2 * y * y ) ); // 2nd iteration, this can be removed

   #ifndef Q3_VM
   #ifdef __linux__
     assert( !isnan(y) ); // bk010122 - FPE?
   #endif
   #endif
   return y;
}
\end{lstlisting}




\subsection{矩阵}

\begin{lstlisting}
struct Matrix{
    ll m[15][15];
    Matrix() { memset(m,0,sizeof m); }
};
Matrix mul(Matrix a, Matrix b){
    Matrix c;
    for(int i = 1; i <= n+2; i++){
        for(int k = 1; k <= n+2; k++)
        if(a.m[i][k]){
            for(int j = 1; j <= n+2; j++)
            if(b.m[k][j]) c.m[i][j] = (c.m[i][j]+a.m[i][k]*b.m[k][j])%mod;
        }
    } return c;
}

Matrix powmul(Matrix a, int k){
    Matrix b;
    for(int i = 1; i <= n+2; i++) b.m[i][i] = 1;
    while(k){
        if(k&1) b = mul(b,a);
        a = mul(a,a); k>>=1;
    } return b;
}
\end{lstlisting}



\subsubsection{带模行列式求解}
\begin{lstlisting}
// `高斯消元法行列式求模,复杂度O($n^3logn$)。`
// `n为行列式大小,计算|mat| \% m`
const int MAXN = 200;
typedef long long ll;
int detMod(int n, int m, int mat[][MAXN]) {
    for (int i = 0; i < n; i++)
        for (int j = 0; j < n; j++) mat[i][j] %= m;
    ll ret = 1;
    for (int i = 0; i < n; i++) {
        for (int j = i + 1; j < n; j++)
            while (mat[j][i] != 0) {
                ll t = mat[i][i] / mat[j][i];
                for (int k = i; k < n; k++) {
                    mat[i][k] = (mat[i][k] - mat[j][k] * t) % m;
                    int s = mat[i][k];
                    mat[i][k] = mat[j][k]; mat[j][k] = s;
                } ret = -ret;
            }
        if (mat[i][i] == 0) return 0;
        ret = ret * mat[i][i] % m;
    } if (ret < 0) ret += m;
    return (int)ret;
}

// `高斯消元法行列式求模。复杂度O($n^3 + n^2logn$)`
// `n为行列式大小,计算|mat| \% m`
// `速度只比O($n^3logn$)的快一些,推荐用另外那个。`
const int MAXN = 200;
typedef long long ll;
int detMod(int n, int m, int mat[][MAXN]) {
    for (int i = 0; i < n; i++)
        for (int j = 0; j < n; j++)mat[i][j] %= m;
    ll ret = 1;
    for (int i = 0; i < n; i++) {
        for (int j = i + 1; j < n; j++) {
            ll x1 = 1, y1 = 0, x2 = 0, y2 = 1, p = i, q = j;
            if (mat[i][i] < 0) {
                x1 = -1; mat[i][i] = -mat[i][i]; ret = -ret;
            } if (mat[j][i] < 0) {
                y2 = -1; mat[j][i] = -mat[j][i]; ret = -ret;
            } while (mat[i][i] != 0 && mat[j][i] != 0) {
                if (mat[i][i] <= mat[j][i]) {
                    int t = mat[j][i] / mat[i][i];
                    mat[j][i] -= mat[i][i] * t;
                    x2 -= x1 * t;
                    y2 -= y1 * t;
                } else {
                    int t = mat[i][i] / mat[j][i];
                    mat[i][i] -= mat[j][i] * t;
                    x1 -= x2 * t;
                    y1 -= y2 * t;
                }
            }
            x1 %= m; y1 %= m;
            x2 %= m; y2 %= m;
            if (mat[i][i] == 0 && mat[j][i] != 0) {
                ret = -ret;
                p = j; q = i;
                mat[i][i] = mat[j][i];
                mat[j][i] = 0;
            }
            for (int k = i + 1; k < n; k++) {
                int s = mat[i][k], t = mat[j][k];
                mat[p][k] = (s * x1 + t * y1) % m;
                mat[q][k] = (s * x2 + t * y2) % m;
            }
        } if (mat[i][i] == 0) return 0;
        ret = ret * mat[i][i] % m;
    }
    if (ret < 0) ret += m;
    return (int)ret;
}
\end{lstlisting}

\subsubsection{高斯消元}
消元部分
\begin{lstlisting}
void Gauss(double a[][N], int n ,int m, int &r, int &c){
    r=c=0;
    for(;r<n&&c<m;r++,c++) {
        int maxi = r;
        for(int i = r+1; i < n; i++)
            if(fabs(a[i][c])>fabs(a[maxi][c])) maxi = i;
        if(maxi!=r)
            for(int i = r; i < m+1; i++) swap(a[maxi][i],a[r][i]);
        if(!a[r][c]){ r--; continue; }
        for(int i = r+1; i < n; i++){
            if(a[i][c]){
                double t = -a[i][r]/a[r][r];
                for(int j = r; j < m+1; j++) a[i][j] += t*a[r][j];
            }
        }
    }
}
// `模方程`
void Gauss(int a[][N], int n, int m, int &r, int &c){
    memset(fre, 0, sizeof fre);
    r = c = cnt = 0;
    for(; r < n && c < m ; r++,c++){
        int maxi = r;
        for(int i = r+1; i < n; i++)
            if(abs(a[maxi][c]) < abs(a[i][c])) maxi = i;
        if(maxi!=r) 
            for(int i = r; i < m+1; i++) swap(a[r][i],a[maxi][i]);
        if(!a[r][c]) { fre[cnt++]=c; r --;continue; }
        for(int i = r +1; i < n; i++){
            if(a[i][c]){
                int x = abs(a[i][c]), y = abs(a[r][c]);
                int LCM = lcm(x,y);
                int tx = LCM/x, ty = LCM/y;
                if(a[r][c]*a[i][c] < 0) ty = -ty;
                for(int j =c ; j < m+1; j++)
                    a[i][j] = (a[i][j]%mod*tx%mod - a[r][j]%mod*ty%mod+mod)%mod;
            }
        }
    }
}
\end{lstlisting}

回代部分
\begin{lstlisting}
void Rewind(double a[][N], double x[], int n, int m, int r, int c){
    for(int i = r-1; i >= 0; i--){
        //`再不改变原方程的情况下计算解`
        double t = a[i][c];
        for(int j = i+1; j < c; j++) t-=a[i][j]*x[j];
        if(a[i][i]) x[i] = t/a[i][i];
    }
}
//`模方程`
int Rewind(int a[][N], int x[N], int n, int m, int r, int c){
    for(int i = r ; i < n; i++) if(a[i][c]) return -1; // `无解`
    if(cnt || r<m) return 1; //`多解`
    //`求解`
    memset(x, 0, sizeof x);
    for(int i = r-1; i >= 0; i--) {
        int temp = a[i][c] % mod;
        for(int j = i+1; j < c; j++) temp -= a[i][j]*x[j];
        //`求逆元的欧几里得`
        int d,xx,yy;
        e_gcd(a[i][i], mod, d, xx, yy);
        xx = (xx%mod+mod)%mod;
        x[i] = ((temp%mod*xx%mod)%mod+mod)%mod;
        if(x[i]<3) x[i] += mod;
    } return 0;
}
\end{lstlisting}

\subsubsection{异或方程组}
\begin{lstlisting}
int xor_guass(int m, int n) {// `A是异或方程组系数矩阵 返回秩`
    int i = 0, j = 0, k, r, u;
    while(i < m && j < n){//`当前正在处理第i个方程,第j个变量`
        r = i;
        for(int k = i; k < m; k++) if(A[k][j]){r = k; break;}
        if(A[r][j]){
             if(r != i) for(k = 0; k <= n; k++) swap(A[r][k], A[i][k]);
             // `消元完成之后第i行的第一个非0列是第j列,且第u>i行的第j列全是0`
            for(u = i + 1; u < m; u++) if(A[u][j])
                for(k = i; k <= n; k++) A[u][k] ^= A[i][k];
            i++;
        } j++;
    } return i; // `返回矩阵的秩即,非自由变元的个数`
}
\end{lstlisting}

\subsubsection{线性基}
\begin{lstlisting}
struct L_B{
	long long d[61],p[61];
	int cnt;
	L_B() {
		memset(d,0,sizeof(d));
		memset(p,0,sizeof(p));
		cnt=0;
	}
	//`插入`
	bool insert(long long val) {
		for (int i=60;i>=0;i--)
		if (val&(1LL<<i)) {
			if (!d[i]) {
				d[i]=val;
				break;
			} val^=d[i];
		} return val>0;
	}
	//`查询最大值`
	long long query_max() {
		long long ret=0;
		for (int i=60;i>=0;i--) if ((ret^d[i])>ret) ret^=d[i];
		return ret;
	}
	//`查询最小值`
	long long query_min() {
		for (int i=0;i<=60;i++) if (d[i]) return d[i];
		return 0;
	}
	//`查找第k小时的重建(将线性基改造成每一位都相互独立)`
	void rebuild() {
		for (int i=60;i>=0;i--) for (int j=i-1;j>=0;j--)
		    if (d[i]&(1LL<<j)) d[i]^=d[j];
		for (int i=0;i<=60;i++) if (d[i]) p[cnt++]=d[i];
	}
	long long kthquery(long long k) {
		int ret=0;
		if (k>=(1LL<<cnt)) return -1;
		for (int i=60;i>=0;i--) if (k&(1LL<<i)) ret^=p[i];
		return ret;
	}
}
//`合并`
L_B merge(const L_B &n1,const L_B &n2) {
	L_B ret=n1;
	for (int i=60;i>=0;i--) if (n2.d[i]) ret.insert(n1.d[i]);
	return ret;
}
\end{lstlisting}


\subsection{大整数}
\begin{lstlisting}
struct BigInteger {
    int len, s[SIZE + 5];

    BigInteger () {
        memset(s, 0, sizeof(s));
        len = 1;
    }
    BigInteger operator = (const char *num) { //`字符串赋值`
        memset(s, 0, sizeof(s));
        len = strlen(num);
        for(int i = 0; i < len; i++) s[i] = num[len - i - 1] - '0';
        return *this;
    }
    BigInteger operator = (const int num) { //int `赋值`
        memset(s, 0, sizeof(s));
        char ss[SIZE + 5];
        sprintf(ss, "%d", num);
        *this = ss;
        return *this;
    }
    BigInteger (int num) {
        *this = num;
    }
    BigInteger (char* num) {
        *this = num;
    }
    string str() const { //`转化成`string
        string res = "";
        for(int i = 0; i < len; i++) res = (char)(s[i] + '0') + res;
        if(res == "") res = "0";
        return res;
    }
    BigInteger clean() {
        while(len > 1 && !s[len - 1]) len--;
        return *this;
    }

    BigInteger operator + (const BigInteger& b) const {
        BigInteger c;
        c.len = 0;
        for(int i = 0, g = 0; g || i < max(len, b.len); i++) {
            int x = g;
            if(i < len) x += s[i];
            if(i < b.len) x += b.s[i];
            c.s[c.len++] = x % 10;
            g = x / 10;
        }
        return c.clean();
    }

    BigInteger operator - (const BigInteger& b) {
        BigInteger c;
        c.len = 0;
        for(int i = 0, g = 0; i < len; i++) {
            int x = s[i] - g;
            if(i < b.len) x -= b.s[i];
            if(x >= 0) g = 0;
            else {
                g = 1;
                x += 10;
            }
            c.s[c.len++] = x;
        }
        return c.clean();
    }

    BigInteger operator * (const int num) const {
        int c = 0, t;
        BigInteger pro;
        for(int i = 0; i < len; ++i) {
            t = s[i] * num + c;
            pro.s[i] = t % 10;
            c = t / 10;
        }
        pro.len = len;
        while(c != 0) {
            pro.s[pro.len++] = c % 10;
            c /= 10;
        }
        return pro.clean();
    }

    BigInteger operator * (const BigInteger& b) const {
        BigInteger c;
        for(int i = 0; i < len; i++) {
            for(int j = 0; j < b.len; j++) {
                c.s[i + j] += s[i] * b.s[j];
                c.s[i + j + 1] += c.s[i + j] / 10;
                c.s[i + j] %= 10;
            }
        }
        c.len = len + b.len + 1;
        return c.clean();
    }

    BigInteger operator / (const BigInteger &b) const {
        BigInteger c, f;
        for(int i = len - 1; i >= 0; --i) {
            f = f * 10;
            f.s[0] = s[i];
            while(f >= b) {
                f = f - b;
                ++c.s[i];
            }
        }
        c.len = len;
        return c.clean();
    }
    //`高精度取模`
    BigInteger operator % (const BigInteger &b) const{
        BigInteger r;
        for(int i = len - 1; i >= 0; --i) {
            r = r * 10;
            r.s[0] = s[i];
            while(r >= b) r = r - b;
        }
        r.len = len;
        return r.clean();
    }

    bool operator < (const BigInteger& b) const {
        if(len != b.len) return len < b.len;
        for(int i = len - 1; i >= 0; i--)
            if(s[i] != b.s[i]) return s[i] < b.s[i];
        return false;
    }
    bool operator > (const BigInteger& b) const {
        return b < *this;
    }
    bool operator <= (const BigInteger& b) const {
        return !(b < *this);
    }
    bool operator == (const BigInteger& b) const {
        return !(b < *this) && !(*this < b);
    }
    bool operator != (const BigInteger &b) const {
        return !(*this == b);
    }
    bool operator >= (const BigInteger &b) const {
        return *this > b || *this == b;
    }
    friend istream & operator >> (istream &in, BigInteger& x) {
        string s;
        in >> s;
        x = s.c_str();
        return in;
    }
    friend ostream & operator << (ostream &out, const BigInteger& x) {
        out << x.str();
        return out;
    }
};
\end{lstlisting}

分数类见红书


\subsection{卷积}

\subsubsection{FFT}
\begin{lstlisting}
namespace FFT {
    #define rep(i,a,b) for(int i=(a);i<=(b);i++)
    const double pi=acos(-1);
    const int maxn=1<<19;
    struct cp {
        double a,b;
        cp(){}
        cp(double _x,double _y){a=_x,b=_y;}
        cp operator +(const cp &o)const{return (cp){a+o.a,b+o.b};}
        cp operator -(const cp &o)const{return (cp){a-o.a,b-o.b};}
        cp operator *(const cp &o)const{return (cp){a*o.a-b*o.b,b*o.a+a*o.b};}
        cp operator *(const double &o)const{return (cp){a*o,b*o};}
        cp operator !()const{return (cp){a,-b};}
    } x[maxn],y[maxn],z[maxn],w[maxn];
    void fft(cp x[],int k,int v) {
        int i,j,l;
        for(i=0,j=0;i<k;i++) {
            if(i>j)swap(x[i],x[j]);
            for(l=k>>1;(j^=l)<l;l>>=1);
        } w[0]=(cp){1,0};
        for(i=2;i<=k;i<<=1) {
            cp g=(cp){cos(2*pi/i),(v?-1:1)*sin(2*pi/i)};
            for(j=(i>>1);j>=0;j-=2)w[j]=w[j>>1];
            for(j=1;j<i>>1;j+=2)w[j]=w[j-1]*g;
            for(j=0;j<k;j+=i) {
                cp *a=x+j,*b=a+(i>>1);
                for(l=0;l<i>>1;l++) {
                    cp o=b[l]*w[l];
                    b[l]=a[l]-o;
                    a[l]=a[l]+o;
                }
            }
        } if(v)for(i=0;i<k;i++)x[i]=(cp){x[i].a/k,x[i].b/k};
    }
    void mul(ll *a,ll *b,ll *c,int l1,int l2) {
        if(l1<128&&l2<128) {
            rep(i,0,l1+l2)a[i]=0;
            rep(i,0,l1)rep(j,0,l2)a[i+j]+=b[i]*c[j];
            return;
        } int K;
        for(K=1;K<=l1+l2;K<<=1);
        rep(i,0,l1)x[i]=cp(b[i],0);
        rep(i,0,l2)y[i]=cp(c[i],0);
        rep(i,l1+1,K)x[i]=cp(0,0);
        rep(i,l2+1,K)y[i]=cp(0,0);
        fft(x,K,0);fft(y,K,0);
        rep(i,0,K)z[i]=x[i]*y[i];
        fft(z,K,1);
        rep(i,0,l1+l2)a[i]=(ll)(z[i].a+0.5);
    }
};
\end{lstlisting}

{\bfseries 典型例题:CodeForces - 528D }

题目大意:给出一个母串和一个模板串,求模板串在母串中的匹配次数。 
匹配时,如果用s[i]匹配t[j],那么只要s[i-k]-s[i+k]中有字母与t[j]相同即可算作匹配成功。其中s[i]表示母串的第i位,t[j]表示模板串的第j位。

我们分别考虑每个字符, 那么s[i+p] = t[i] 那么我们反转母串就是卷积。a[j+i] - b[i] 反转 -> a[n-1-j-i] - b[i] 那么 下标和就是 n - 1 - j

我们取卷积后前n个元素就是匹配的个数,所有字符计算一下求和判断是不是|T|即可。

如果字符匹配不是一一配对,也就是说可以 a[j+i+p] - b[i] (p <= k) 那么我们只要用树状数组取标记是否有贡献即可。

\begin{lstlisting}
#include <bits/stdc++.h>

using namespace std;
const int N = 1e6+9;

namespace fft{
#define MAX 500000
    struct complx{
        double real, img;
        inline complx(){ real = img = 0.0; }
        inline complx(double x){ real = x, img = 0.0; }
        inline complx(double x, double y){ real = x, img = y; }
        inline void operator += (complx &other){ real += other.real, img += other.img; }
        inline void operator -= (complx &other){ real -= other.real, img -= other.img; }
        inline complx operator + (complx &other){ return complx(real + other.real, img + other.img); }
        inline complx operator - (complx &other){ return complx(real - other.real, img - other.img); }
        inline complx operator * (complx& other){ return complx((real * other.real) - (img * other.img), (real * other.img) + (img * other.real)); }
    };
    int rev[MAX];
    complx P[MAX >> 1], P1[MAX], P2[MAX];
    void FFT(complx *ar, int n, int inv){
        int i, j, k, l, len, len2;
        const double p = 4.0 * inv * acos(0.0);
        for (i = 0, l = __builtin_ctz(n) - 1; i < n; i++){
            rev[i] = rev[i >> 1] >> 1 |((i & 1) << l);
            if(i < rev[i]) swap(ar[i], ar[rev[i]]);
        }
        for (len = 2; len <= n; len <<= 1){
            len2 = len >> 1;
            double theta = p / len;
            complx mul(cos(theta), sin(theta));
            for (i = 1, P[0] = complx(1, 0); i < len2; i++) P[i] = (P[i - 1] * mul);
            for (i = 0; i < n; i += len){
                complx t, *x = ar + i, *y = ar + i + len2, *l = ar + i + len2, *z = P;
                for (; x != l; x++, y++, z++){
                    t = (*y) * (*z), *y = *x - t;
                    *x += t;
                }
            }
        }
        if (inv == -1){
            double tmp = 1.0 / n;
            for (i = 0; i < n; i++) ar[i].real *= tmp;
        }
    }
    void cmplx_conv(int n, complx* A, complx* B){
        int m = 1 << (32 - __builtin_clz(n) - (__builtin_popcount(n) == 1));
        for (int i = n; i < m; i++) A[i] = B[i] = complx(0);
        FFT(A, m, 1), FFT(B, m, 1);
        for (int i = 0; i < m; i++) A[i] = A[i] * B[i];
        FFT(A, m, -1);
    }
    void conv(int n, int* A, int* B){
        for (int i = 0; i < n; i++) P1[i] = complx(A[i], 0.0), P2[i] = complx(B[i], 0.0);
        cmplx_conv(n, P1, P2);
        for (int i = 0; i < n; i++) A[i] = floor(P1[i].real + 0.5);
    }
}

char S[N], T[N];
const char DNA[] = "ACGT";
int n, m, k, a[N], b[N], tree[N], cont[4][N];
inline int lb(int i) {return i&(-i);}
void update(int p, int v) { for (int i = p; i <= n; i += lb(i)) { tree[i] += v; } }

void update(int l, int r, int v) {
    if(l > r) return ;
    update(l, v); update(r + 1, -v);
}

int query(int p) {
    int ans = 0;
    for (int i = p; i; i -= lb(i)) { ans += tree[i]; } 
    return  ans;
}

int main() {
    scanf("%d%d%d%s%s", &n, &m, &k, S, T);
    for (int i = 0; i < 4; i++) {
        memset(a, 0, sizeof a); memset(b, 0, sizeof b); memset(tree, 0, sizeof tree);
        for (int j = 0; j < m; j++) a[j] = (T[j] == DNA[i]);
        for (int j = 0; j < n; j++) 
            if (S[j] == DNA[i])
                update(max(0, j - k) + 1, min(j+k+1, n), 1);
        for (int j = 0; j < n; j++) if(query(j+1)) b[j] = 1;
        reverse(b, b+n); fft::conv(n, a, b);
        for (int j = 0; j < n; j++) cont[i][j] = a[n - j - 1];
    } int ans = 0;
    for (int i = 0; (i+m) <= n; i++) 
        if(cont[0][i] + cont[1][i] + cont[2][i] + cont[3][i] == m) ans ++;
    return printf("%d\n", ans), 0;
}
\end{lstlisting}

\subsubsection{FWT}

\begin{lstlisting}
void FWT(ll *a,int opt) {
    for(int i=1;i<n;i<<=1)
        for(int p=i<<1,j=0;j<n;j+=p)
            for(int k=0;k<i;++k)
            {
                ll X=a[j+k],Y=a[i+j+k];
                a[j+k]=(X+Y)%mod;a[i+j+k]=(X+mod-Y)%mod;
                if(opt==-1)a[j+k]=1ll*a[j+k]*inv2%mod,a[i+j+k]=1ll*a[i+j+k]*inv2%mod;
            }
}
\end{lstlisting}

典型例题:51nod 多校第九场A
A[i][j] = a[i xor j]

For example, when n = 4, the equations look like

A[0][0]*x[0] + A[0][1]*x[1] + A[0][2]*x[2] + A[0][3]*x[3] = b[0] (mod p)

A[1][0]*x[0] + A[1][1]*x[1] + A[1][2]*x[2] + A[1][3]*x[3] = b[1] (mod p)

A[2][0]*x[0] + A[2][1]*x[1] + A[2][2]*x[2] + A[2][3]*x[3] = b[2] (mod p)

A[3][0]*x[0] + A[3][1]*x[1] + A[3][2]*x[2] + A[3][3]*x[3] = b[3] (mod p)

\begin{lstlisting}
int main() {
    scanf("%d", &n); inv2 = qp(2);
    for (int i = 0; i < n; i++) {
        scanf("%lld", &a[i]);
    } for (int i = 0; i < n; i++) {
        scanf("%lld", &b[i]);
    } FWT(a, 1); FWT(b, 1);
    for (int i = 0; i < n; i++) {
        b[i] = (b[i]*qp(a[i])) % mod;
    } FWT(b, -1);
    for (int i = 0; i < n; i++) printf("%lld\n", gm(b[i]));
}
\end{lstlisting}

\subsection{01分数规划}
给出 n对 $a_i, b_i$, 求最大的$\sum \frac{a_i}{b_i}$ ,最多可以删除k对ai,bi,
除了二分以外,可以使用01分数规划迭代,可以解决最优比例生成树

\begin{lstlisting}
const int maxn = 1e3+10;
using namespace std;
int n,k;
struct node{ int a,b; } val[maxn];
double p;
int cmp(node x, node y){ return x.a - (p * x.b) > y.a - (p * y.b); }

bool check(double mid){
    p = mid;
    sort(val, val + n, cmp);
    double total_a = 0,total_b = 0;
    for(int i = 0; i < n - k; i++){
        total_a += val[i].a;
        total_b += val[i].b;
    } return (total_a/total_b) > mid;
}

int main(){
    while(~scanf("%d%d",&n,&k) && n){
        for(int i = 0; i < n; i++) scanf("%d",&val[i].a);
        for(int i = 0; i < n; i++) scanf("%d",&val[i].b);
        double l = 0, r = 1;
        double mid;
        for(int i = 0; i < 200; i++){
            mid = (l+r)/2;
            if(check(mid)) l = mid;
            else r = mid;
        } printf("%d\n",(int)round(mid*100));
    } return 0;
}
\end{lstlisting}

\subsection{三分}

\begin{lstlisting}
for(int i = 0; i < 200; i++){
    mid1 = l + (r-l)/3;
    mid2 = r - (r-l)/3;
    if(cal(mid1) < cal(mid2)) r = mid2;
    else l = mid1;
}
\end{lstlisting}


\subsubsection{黄金比例三分}

\begin{lstlisting}
const double phi=(sqrt(5.0)-1)/2;
double l = 0, r = 2*PI, mid1, mid2,f1=0,f2=100;
int left = 0,right = 0;
for(int i = 0; i < 35 && fabs(l-r)> 1e-5; i++){
  if(!left)mid1 = r - (r-l)*phi, f1 = cal(mid1);
  if(!right)mid2 = l + (r-l)*phi, f2 = cal(mid2);
  if(f1 < f2) r = mid2; mid2 = mid1;left = 0; right = 1; f2 = f1;
  else l = mid1; mid1 = mid2; right = 0; left = 1; f1 = f2;
}
\end{lstlisting}

\subsection{模拟退火}

\subsubsection{费马点}
给n个点,找出一个点,使这个点到其他所有点的距离之和最小,也就是求费马点。
\begin{lstlisting}
#include <bits/stdc++.h>
#define N 1005
#define eps 1e-8     //`搜索停止条件阀值`
#define INF 1e99
#define delta 0.98   //`温度下降速度`
#define T 100        //`初始温度`

using namespace std;

int dx[4] = {0, 0, -1, 1};
int dy[4] = {-1, 1, 0, 0};  //`上下左右四个方向`

struct Point { double x, y; }p[N];

double dist(Point A, Point B) {
	return sqrt((A.x - B.x) * (A.x - B.x) + (A.y - B.y) * (A.y - B.y));
}

double GetSum(Point p[], int n, Point t) {
	double ans = 0;
	while(n--) ans += dist(p[n], t);
	return ans;
}

double Search(Point p[], int n) {
	Point s = p[0];    //`随机初始化一个点开始搜索`
	double t = T;      //`初始化温度`
	double ans = INF;  //`初始答案值`
	while(t > eps) {
		bool flag = 1;
		while(flag) {
			flag = 0;
		    for(int i = 0; i < 4; i++) {   //`上下左右四个方向`
			    Point z;
			    z.x = s.x + dx[i] * t;
			    z.y = s.y + dy[i] * t;
			    double tp = GetSum(p, n, z);
		        if(ans > tp) { ans = tp; s = z; flag = 1; }
		    }
		} t *= delta;
	} return ans;
}

int main() {
	int n;
	while(scanf("%d", &n) != EOF) {
		for(int i = 0; i < n; i++) scanf("%f %f", &p[i].x, &p[i].y);
		printf("%.0lf\n", Search(p, n));
	} return 0;
}
\end{lstlisting}


\subsubsection{距离直线簇最近点}

平面上给定n条线段,找出一个点,使这个点到这n条线段的距离和最小。

\begin{lstlisting}
struct Point { double x, y; }s[N], t[N];
double cross(Point A, Point B, Point C) { return (B.x - A.x) * (C.y - A.y) - (B.y - A.y) * (C.x - A.x); }
double multi(Point A, Point B, Point C) { return (B.x - A.x) * (C.x - A.x) + (B.y - A.y) * (C.y - A.y); }
double dist(Point A, Point B) { return sqrt((A.x - B.x) * (A.x - B.x) + (A.y - B.y) * (A.y - B.y)); }

double GetDist(Point A, Point B, Point C) {
	if(dist(A, B) < eps) return dist(B, C);
	if(multi(A, B, C) < -eps) return dist(A, C);
	if(multi(B, A, C) < -eps) return dist(B, C);
	return fabs(cross(A, B, C) / dist(A, B));
}

double GetSum(Point s[], Point t[], int n, Point o) {
	double ans = 0;
	while(n--) ans += GetDist(s[n], t[n], o);
	return ans;
}
\end{lstlisting}

\subsubsection{包住所有点的球}

\begin{lstlisting}
double Search(Point p[], int n) {
	Point s = p[0];
	double t = T;
	double ans = INF;
	while(t > eps) {
		int k = 0;
		for(int i = 0; i < n; i++)
			if(dist(s, p[i]) > dist(s, p[k])) k = i;
		double d = dist(s, p[k]);
		ans = min(ans, d);
		s.x += (p[k].x - s.x) / d * t;
		s.y += (p[k].y - s.y) / d * t;
		s.z += (p[k].z - s.z) / d * t;
		t *= delta;
	} return ans;
}
\end{lstlisting}


\subsubsection{函数最值问题}

\begin{lstlisting}
int Judge(double x, double y) {
	if(fabs(x - y) < eps) return 0;
	return x < y ? -1 : 1;
}

double Random() {
	double x = rand() * 1.0 / RAND_MAX;
	if(rand() & 1) x *= -1;
	return x;
}

void Init() {
	for(int i = 0; i < ITERS; i++) x[i] = fabs(Random()) * 100;
}

double SA(double y) {
	double ans = INF;
	double t = T;
	while(t > eps) {
		for(int i = 0; i < ITERS; i++) {
			double tmp = F(x[i], y);
			for(int j = 0; j < ITERS; j++) {
				double _x = x[i] + Random() * t;
				if(Judge(_x, 0) >= 0 && Judge(_x, 100) <= 0) {
					double f = F(_x, y);
					if(tmp > f) x[i] = _x;
				}
			} 
		} t *= delta;
	}
	for(int i = 0; i < ITERS; i++) ans = min(ans, F(x[i], y));
	return ans;
}

srand(time(NULL));
Init();
\end{lstlisting}


