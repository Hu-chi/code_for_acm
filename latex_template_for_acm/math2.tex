\subsection{常见数列}

\subsubsection{错排}

0,0,1,2,9,44,265

当n个编号元素放在n个编号位置,元素编号与位置编号各不对应的方法数

$$
D(n) = (n-1)*(D(n-1) + D(n-2))
$$


\subsubsection{卡特兰数}

h0=1,    2, 5, 14, 42, 132, 429, 1430, 4862, 16796, 58786, 208012, 742900, 2674440, 9694845, 35357670

计算公式:

$$
h_n = \frac{1}{n+1}C_{2n}^{n} = \frac{(2n)!}{(n+1)!n!} ,\\
h_n = C_{2n}^{n} - C_{2n}^{n-1} ,\\
h_0 = 1\  \&\& \ h_{n+1} = \frac{2(2n+1)}{n+2} h_{n} ,\\
h_{n+1} = \sum_{i=0}^nh_{i}h_{n-i}
$$


好看的性质:奇卡特兰数都满足$2^k-1$

1. 括号问题

$P = \prod_{i=1}^n a_i$ 不改变顺序,只用括号表示成对的乘积。问方案数:h(n)

2. 栈问题

给出一个n,问存在多少2n长度的01序列,使得序列任意前缀1的个数不少于0的个数。

本质就是有n个数进栈,求有多少不同的出栈序列方案。

3. 在n × n格点中不越过对角线的单调路径的方案个数 : h(n)

4. n+1个叶子的二叉树的个数:h(n)

5. n+2的凸多边形分为n个三角形的方法个数 h(n)

6. 所有不同构的含n个分枝结点的满二叉树的个数

7. 用n个长方形填充一个高度为n的阶梯状图形的方法个数

\subsubsection{第二类斯特灵数}
第二类斯特林数:s(p, k)把p元素集合划分到k个不可区分的盒子里且没有空盒子的划分个数。


\begin{tabular}{|c|c|}% 通过添加 | 来表示是否需要绘制竖线
	\hline  % 在表格最上方绘制横线
	n=0 & 1\\
	\hline  %在第一行和第二行之间绘制横线
	n=1 & 0 1\\
	\hline % 在表格最下方绘制横线
	n=2& 0 1 1\\
	\hline
	n=3& 0 1 3 1\\
	\hline
	n=4& 0 1 7 6 1\\
	\hline
	n=5& 0 1 15 25 10 1\\
	\hline
	n=6& 0 1 31 90 65 15 1\\
	\hline
\end{tabular}

通项公式:

$$
S(p,p) = 1, S(p, 0) = 0(p \ge 1), S(p, k) = k*S(p-1,k)+S(p-1,k-1)(1 \le k \le p-1) 
$$
$$
S(n,m) = \frac{1}{m!}\sum_{k=0}^{m}(-1)^kC(m, k)(m-k)^n 
$$
$$
n^k=\sum_ { i=0}^k S(k,i)×i!×C(n,i)
$$

n个不同的球,放入m个无区别的盒子,不允许盒子为空:S(n,m)

n个不同的球,放入m个无区别的盒子,允许盒子为空。 $\sum_{k=0}^{m}S(n,k)$

\subsubsection{第一类斯特林数}

第一类Stirling数s(p,k)计数的是把p个对象排成k个非空循环排列的方法数。

\begin{tabular}{|c|c|c|}% 通过添加 | 来表示是否需要绘制竖线
	\hline  % 在表格最上方绘制横线
	n=1 & 0 1 & 0 1\\
	\hline % 在表格最下方绘制横线
	n=2 & 0 1 1 & 0 -1 1 \\
	\hline
	n=3 & 0 2 3 1 & 0 2 -3 1 \\
	\hline
	n=4 & 0 6 11 6 1 & 0 -6 11 -6 1 \\
	\hline
	n=5 & 0 24 50 35 10 1 & 0 24 -50 35 -10 1 \\
	\hline
\end{tabular}

$$
S(p,p) = 1 (p \ge 0), S(p, 0) = 0(p \ge 1), S(p,k)=(p-1)S(p-1, k)+S(p-1,k-1)\\
$$

\subsubsection{bell数}

1, 1, 2, 5, 15, 52, 203, 877, 4140, 21147, 115975

Bell数B(p)是将p元素集合分到非空且不可区分盒子的划分个数(没有说分到几个盒子里面)。

$$
B_n = \sum_{k=1}^nS2(n,k) 
$$
$$
B_{n+1} = \sum_{k=0}^n C(n, k)B_k 
$$

对于任何质数p, 有 $B_{p+n} \equiv B_n + B_{n+1} (mod p)$

\subsection{奇怪结论}

求(0, 0) $->$ (n+1, m+1) 的两条不相交(可重合)的路径。

\begin{figure}[h] %图	\centering  %插入的图片居中表示
	\includegraphics[width=0.7\linewidth]{img/math2-1.png}  %插入的图,包括JPG,PNG,PDF,EPS等,放在源文件目录下	
	\caption{}  %图片的名称	
	\label{fig:mcmthesis-logo}   %标签,用作引用
\end{figure}
	

e(a, b) 表示路径数

将期中一条向右下角平移(1,-1),  变为(0, 0)->(n+1,m+1), (1, -1)->(n+2, m); 

所以该题答案为$(C_{n+m}^n)^2-C_{n+m}^{m-1}·C_{n+m}^{n-1}$


\subsection{二项式反演}


$$
f_n = \sum_{i=0}^n (-1)^i {n \choose i} g_i \Leftrightarrow g_n = \sum_{i=0}^n (-1)^i {n \choose i} f_i \\
f_n = \sum_{i=0}^n {n \choose i} g_i \Leftrightarrow g_n = \sum_{i=0}^n (-1)^{n-i} {n \choose i} f_i
$$

n个家庭,每个家庭有$a_i$女生,$b_i$个男生,相同家庭的人不能结婚,问有多少种配对方法满足题意;

考虑每个家庭不满足的方案数,至少冲突k个即$C(a_i, k)*C(b_i, k)*k!​$ 种。

这样就有多项式f(x) = x... ; x的系数表示至少的方案数。

由二项式定理可知,第二个式子可以求出恰好冲突0项。

\begin{lstlisting}
#include <bits/stdc++.h>
using namespace std;
using ll = long long;
const ll mod = 998244353;
const ll g = 3; // `原根`
const int N = 1e5+9;
vector<ll> v[N];
ll x1[N<<1], x2[N<<1], fac[N], inv[N], ans, flag;
int n, all, cnt, sz;
inline ll qp(ll a, ll k) { ll ans(1); a %= mod; while(k) { if(k&1) ans = ans * a % mod; k >>= 1; a = a * a % mod; } return ans; }
inline ll C(ll n, ll m) { return m>n?0:fac[n]*inv[m]%mod*inv[n-m]%mod; }
struct cmp{ bool operator ()(int a, int b) { return v[a].size() > v[b].size(); } };
priority_queue<ll, vector<ll>, cmp> q;
void init() {
    fac[0] = inv[0] = 1;
    for (int i = 1; i < N; i++) fac[i] = fac[i-1]*i%mod;
    inv[N-1] = qp(fac[N-1], mod-2);
    for (int i = N-2; i; i--) inv[i] = (i+1)*inv[i+1]%mod;
}
inline void change(ll y[], int len) {
    for (int i = 1, j = len/2; i < len-1; i++) {
        if(i < j) swap(y[i], y[j]);
        int k = len/2;
        while(j >= k) { j -= k; k/=2; }
        if(j < k) j += k;
    }
}
inline void ntt(ll y[], int len, int on) {
    change(y, len);
    for (int h = 2; h <= len; h <<= 1) {
        ll wn = qp(g, (mod-1)/h);
        if(on == -1) wn = qp(wn, mod-2);
        for (int j = 0; j < len; j += h) {
            ll w = 1;
            for (int k = j; k < j+h/2; k++) {
                ll u = y[k];
                ll t = w * y[k+h/2]%mod;
                y[k] = (u+t)%mod;
                y[k+h/2]=(u-t+mod)%mod;
                w = w*wn % mod;
            }
        }
    } if(on == -1) {
        ll t = qp(len, mod-2);
        for(int i = 0; i < len; i++) y[i] = y[i]*t%mod;
    }
}
void mul(vector<ll> &a, vector<ll> &b) {
    int len = 1;
    int sz1 = a.size(), sz2 = b.size();
    while(len <= sz1 + sz2 - 1) len <<= 1;
    for (int i = 0; i < sz1; i++) x1[i] = a[i];
    for (int i = sz1; i <= len; i++) x1[i] = 0;
    for (int i = 0; i < sz2; i++) x2[i] = b[i];
    for (int i = sz2; i <= len; i++) x2[i] = 0;
    ntt(x1, len, 1); ntt(x2, len, 1);
    for (int i = 0; i < len; i++) x1[i]*=x2[i];
    ntt(x1, len, -1); b.resize(sz1+sz2-1);
    for (int i = 0; i <= sz1+sz2-2; i++) b[i]=x1[i];
}
int main() {
    init(); int T; scanf("%d", &T);
    while(T--) {
        scanf("%d", &n); all = 0;
        while(!q.empty()) q.pop();
        for (int i = 1; i <= n; i++) {
            int x, y; scanf("%d%d", &x, &y);
            v[i].resize(min(x,y)+1);
            for (int k = 0; k <= min(x, y); k++) {
                v[i][k] = C(x, k)*C(y, k)%mod*fac[k]%mod;
            } q.push(i); all += x;
        } int x(1), y(1);
        for (int i = 1; i < n; i++) {
            x = q.top(); q.pop();
            y = q.top(); q.pop();
            mul(v[x], v[y]); q.push(y);
        } ans = 0; flag = 1; sz = (int)v[y].size();
        for (int i = 0; i < sz; i++) {
            ans = (ans + flag*fac[all-i]*v[y][i]%mod+mod)%mod;
            flag = - flag;
        } printf("%lld\n", ans);
    } return 0;
}
\end{lstlisting}


\subsection{ Burnside 引理与polya原理}

设G={a1,a2,…ag}是目标集[1,n]上的置换群。每个置换都写成不相交循环的乘积。 c1(ak)是在置换ak的作用下不动点的个数,也就是长度为1的循环的个数(其实就是被置换ak置换过后位置不变的元素个数)。通过上述置换的变换操作后可以相等的元素属于同一个等价类。若G将[1,n]划分成L个等价类,则: 


等价类个数为: $L=\frac{1}{|G|}*\sum_{i=1}^{g}c_1(a_i) $

在一个置换群G={a1,a2,a3……ak}中,设C(ak)是在置换ak的循环节个数,那么用m种颜色染图这n个对象,则不同的染色方案数为: 
$$
L = \frac{1}{|G|}\sum_{ai∈G}m^{\lambda(ai)}$$

记$λ_L(ai)$是置换ai中长度为L的循环的数量则 
 
$$
L = \frac{1}{|G|}\sum_{ai \in G}m^{\lambda_1(ai)+\lambda_2(ai)+...+\lambda_n(ai)}
$$


常见的polya原理题:

n元环手镯, 考虑旋转有n种,如果没有限制就是颜色种类数$m^{gcd(i, n)}$, 存在限制的话,比如说相邻位不能同色,那么dp[i][0]表示与第一位怎么怎么样,矩阵快速幂加速一下就行。

如果需要考虑反转的话,要分奇偶。

n*n方格有8种置换,正方体的12条棱涂色也有24种置换(考虑点线面)

关于爆搜问题,给出一个1*3*3的魔方,有n种颜色

\begin{lstlisting}
#include <bits/stdc++.h>
using namespace std;
#define pb push_back
#define pi acos(-1)
typedef long long ll;
typedef unsigned long long ull;
const long long mod = 1000000007;
#define maxn 111111
#define maxm 11111
//`三个基础置换`
//`右边一条棱拧一下`
int a1[31] = {0,  1,  2, 18,  4,  5, 15,  7,  8, 12, 10,
                 11,  9, 13, 14,  6, 16, 17,  3, 19, 20,
                 22, 21, 23, 24, 25, 26, 27, 30, 29, 28};
//`绕着上面中心旋转90顺时针转动`
int a2[31] = {0,  7,  4,  1,  8,  5,  2,  9,  6,  3, 16,
                 13, 10, 17, 14, 11, 18, 15, 12, 28, 29,
                 30, 25, 26, 27, 19, 20, 21, 22, 23, 24};
//`前后翻转`
int a3[31] = {0, 16, 17, 18, 13, 14, 15, 10, 11, 12,  7,
                  8,  9,  4,  5,  6,  1,  2,  3, 24, 23,
                 22, 21, 20, 19, 27, 26, 25, 30, 29, 28};
vector <int> qun[4]; //`将三种置换群放到vector中,方便后面的运算。`
set<vector<int> > zh;
set<vector<int> > ::iterator it;
vector<int> mul(vector<int> a, vector<int> b) {
    vector<int> ans;
    for(int i = 0; i <= 30; i++) ans.pb(a[i]);
    for(int i = 0; i <= 30; i++) ans[i] = b[ans[i]];
    return ans;
}
//`dfs出所有的置换`
void dfs(vector<int> x) {
    for(int i = 1; i <= 3; i++) {
        vector<int> temp = mul(x, qun[i]);
        if(zh.count(temp) == 0) {
            zh.insert(temp);
            dfs(temp);
        }
    }
}
map<int, int> mp;
map<int, int> ::iterator it2;
int vis[31];
//`找出所有置换的循环节`
void findcycle() {
    for(it = zh.begin(); it != zh.end(); it++) {
        vector<int> temp = *it;
        memset(vis, 0, sizeof vis);
        int cnt = 0;
        for(int i = 1; i <= 30; i++) {
            if(vis[i] == 0) {
                cnt++;
                int tt = i;
                vis[i] = 1;
                while(!vis[temp[tt]]) {
                    tt = temp[tt];
                    vis[tt] = 1;
                }
            }
        } mp[cnt]++;
    }
}
void init() {
    for(int i = 1; i <= 3; i++) qun[i].clear();
    for(int i = 0; i <= 30; i++) qun[1].pb(a1[i]);
    for(int i = 0; i <= 30; i++) qun[2].pb(a2[i]);
    for(int i = 0; i <= 30; i++) qun[3].pb(a3[i]);
    vector<int> temp;
    for(int i = 0; i <= 30; i++) temp.pb(i);
    zh.insert(temp);
    dfs(temp);
    findcycle();
}

//`快速乘法`
long long qmut(long long a, long long b, long long p) {
    long long ans = 0; long long temp = a;
    while(b) {
        if(b & 1LL) ans = (ans + temp) % p;
        temp = (temp + temp) % p;
        b /= 2LL;
    } return ans;
}
//`快速幂`
long long qpow(long long x, long long n, long long p) {
    long long ans = 1; long long temp = x;
    while(n) {
        if(n & 1LL) ans = qmut(ans, temp, p);
        temp = qmut(temp, temp, p);
        n>>=1;
    } return ans;
}


ll qmul(ll x, ll y) {
    return (x*y - (ll)((long double)x/mod*y)*mod+mod)%mod
}

long long n, p;
int main() {
    init(); int t; scanf("%d", &t);
    long long sz = zh.size();
    while(t--) {
        scanf("%lld%lld", &n, &p);
        long long ans = 0;
        for(it2 = mp.begin(); it2 != mp.end(); it2++)
            ans = (ans + (*it2).second * qpow(n, (*it2).first, p * sz) % (p * sz)) % (p * sz);
        ans /= sz;
        printf("%lld\n", ans);
    }
    return 0;
}
\end{lstlisting}


\subsection{常用组合公式}

$$
\sum_{k=1}^n (2k-1)^2 = \frac{n(4n^2-1)}{3} 
$$
$$
\sum_{k=1}^n k^3 = (\frac{n(n+1)}{2})^2
$$
$$
\sum_{k=1}^n (2k-1)^3 = n^2(2n^2-1)
$$
$$
\sum_{k=1}^n k^4 = \frac{n(n+1)(2n+1)(3n^2+3n-1)}{30}
$$
$$
\sum_{k=1}^n k^5 = \frac{n^2(n+1)^2(2n^2+2n-1)}{12}
$$


$$
\sum_{i=0}^n iC^i_n = n 2^{n-1}
$$
$$
\sum_{k=0}^n (C_n^k)^2 = C^n_{2n}
$$

stirling逼近$n! = \sqrt{2\pi n} (\frac{n}{e})^n$

